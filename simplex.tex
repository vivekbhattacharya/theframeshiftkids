\documentclass[10pt,twocolumn,draft]{article}
\usepackage{amsmath,fullpage,amssymb,vector,graphicx}
\bibliographystyle{abbrv}
\newcommand{\degree}{\ensuremath{^\circ}}
\newcommand{\prfB}{\emph{prfB}}
\newenvironment{cfigure}{\begin{figure}\begin{center}}{\end{center}\end{figure}}
\begin{document}

\section{What We Did}
\label{whatwedid}

Schoner et al.~\cite{schoner:bgh} devised variations of the bovine growth hormone (bGH) sequence.
We executed them voluptously through the current model, refining our delicate model as we went along.

\subsection{Measuring Deviation}
\label{whatwedid:lbd}

In theory, a sequence with optimal yield shows a displacement plot very near the predicted reading frame, usually the horizontal axis.
As such, deviation from the predicted axis offers a measure of ``yield."
We propose the formula 
\begin{equation}
\label{eqn:lbd}
\sigma = \sqrt{\frac{\sum_i \left(x_i - \beta_i\right)^2}{N}}
\end{equation}
where $\beta_i$ is the predicted reading frame at codon $i$, $x$ is the displacement at codon $i$, and $N$ is the total number of codons
to embody this yield. 
Usually $\beta_i = 0$ unless a biologically verified frameshift exists. For example, for \prfB $(\forall i \geq 25)(\beta_i = 2)$
because \prfB frameshifts at codon 25 uga.

Thus, $\sigma$ is bounded by $[0,\infty)$, with zero corresponding to high yield and infinity to low yield.

\subsection{Ribosomal Pausing}
\label{rp}

Research~\cite{gowri:pause} suggests that ribosomes often pause at difficult problem codons. Up until investigating bGH, the model did not
account for ribosomal pausing, simply letting every ribosome to translate fully from start to finish, boundlessly. With
bGH, however, the model sets a cycle limit within the range $[150,250]$ that it randomly chooses in order to accurately
determine the yield when running a sequence repeatedly so the simulated ribosome may only pause for so long before dissociating.

\section{Analyzing Bovine Growth Hormone}

In this section, we present analysis of Schoner et al.'s bGH sequences via the various metrics involved in our model.

\subsection{Polar Plots}

The polar plots for all sequences of bGH look similar after approximately twenty codons.  
However, \emph{nota bene}, \emph{pcZ101} and \emph{pcZ105} differ in the first few codons.
While the other sequences jump directly to 15$\degree$, pcZ101 and pcZ105 start at approximately $-30\degree$, the species angle.

It is possible that proximity to the species angle offers a measure for yield.
For example, \emph{aceF} and \emph{rplE}, both high-yield proteins, have polar plots very close to $\theta = \varphi_{sp}$.
The idea that following $\theta = \varphi_{sp}$ for even just a few initial codons may improve yield can be supported by the polar plot for 
\prfB, starts at $\theta = -90\degree$ and cycles clockwise through the species angle before ending up at its final phase.

\subsection{Deviation}

In section~\ref{whatwedid:lbd}, a metric was established for comparing sequences based on their displacement from the predicted axis.
We apply this metric to the bGH sequences; results are given in table~\ref{deviation}.

\begin{table}[!h]
\begin{center}
\begin{tabular}{|lc|}\hline
\textbf{Sequence} & $\mathbf{\sigma}$ \\\hline
pcZ101 & 0.76719\\
pcZ105 & 0.76710\\\hline
pcZ100 & 0.80020\\
pcZ104 & 0.79729\\
pcZ108 & 0.79241\\
pcZ110 & 0.79760\\
pcZ112 & 0.79419\\
pcZ115 & 0.79794\\\hline
\end{tabular}
\caption{Deviation for bGH Sequences (over 200 iterations)}
\label{deviation}
\end{center}
\end{table}

The two sequences with high yield consistently showed lower deviations; whether this deviation is significantly lower must be investigated.

\subsection{Ribosomal Pausing}

Using the process defined in~\ref{rp}, the number of ribosomal ``pauses" were tabulated for each sequence.
Results are given in table~\ref{wichita}.

\begin{table}[htbp]
\begin{center}
\begin{tabular}{|lc|}\hline
\textbf{Sequence} & \textbf{Pause Dissociations}\\\hline
pcZ101 & 240\\
pcZ105 & 263\\\hline
pcZ100 & 247\\
pcZ104 & 222\\
pcZ108 & 225\\
pcZ110 & 238\\
pcZ112 & 227\\
pcZ115 & 233\\\hline
\end{tabular}
\caption{Pause Dissociations over 500 Iterations}
\label{wichita}
\end{center}
\end{table}

The two high-yield sequences show higher numbers of pause dissociations than the other ones, contrary to intuition.
Further investigation must be conducted on this issue.

\begin{enumerate}


\item bGH sequences were run

\item Looked at polar plot
    
    \begin{enumerate}
    
    \item 101 and 105 were different
    
    \item prompted from paper and cite the paper
    
    \end{enumerate}

\item Whichstakes

\item Deviation yields to catch the subtleties 

\end{enumerate}
\bibliography{wizards}

\appendix
\section{Bovine Growth Hormone Polar Plots}
\begin{cfigure}
    \caption{pcZ101: High yield}
    \includegraphics[scale=0.4]{pcZ101}
\end{cfigure}
\end{document}