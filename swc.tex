\documentclass{article}
\usepackage{candor}
\usepackage{setspace}
\doublespace
\begin{document}
\section{Introduction}

\section{Description of the Model}

\subsection{Free Energy}
\label{freeenergy}

% Find correct citations for the next paragraph.

Free energy in a cell drives ribosomal translation~\cite{starmer}.
The origin of the free energy comes from hybridization between two sequences of RNA.

As research suggests that the 16S tail of the ribosome hybridizes with the mRNA~\cite{weiss}, 
we can calculate free energy values for the interaction between the two strands of RNA.
Someone et al.~\cite{freier} propose a thermodynamic model for calculation of such free energy values; 
they model the interaction in terms of \emph{doublets}, or pairs of consecutive nucleotides.
Using this model, Freier et al.~\cite{freier} numerically calculate the free energy available upon hybridization for all permutations of RNA doublets.

We use Freier et al.'s values to calculate a free energy value at each nucleotide.  
The 13-base 16S tail of \ecoli\ is first hybridized with the first 13 bases of the sequence 
(for a gene, usually a 12-base leader sequence and the first base of the start codon) to determine
the free energy associated with the first nucleotide.
We can calculate the free energy associated with each subsequent nucleotide by shifting the tail by one base pair.
Since any free energy value greater than zero represent binding that would only take place if energy were added, 
we set such values to zero.

\subsection{Deterministic Model}

Ponnala et al.~\cite{lalit:mechanics} assumes a sinusoidal model for free energy, whence one can calculate
the magnitude and phase of ribosomal translation using arctan given the free energy signals once the memory model
also proposed by Ponnala et al. stores those values in a phaser memory structure,
where memory here is biologically equivalent to the displacement from the reading frame of the ribosome performing the translation
and the registers represent a snapshot of the free energy values at a given codon during translation. The magnitude and phase
of the ribosomal translation here represents the phaser Ponnala uses to structure the register contents and are used to generate
polar plots.

\subsubsection{Displacement}

Ponnala et al.~\cite{lalit:embs} represents the cummulative phasor at codon $k$ as $\vec{V}_k = Me^{j\theta_k}$.  
They then differentiate this vector to derive a vector $\vec{D}_k$ for instantaneous energy, interpreting this
vector as a force that acts on the ribosome to keep the mRNA in a given reading frame.

The length of time that the force acts on the ribosome is dependent on the tRNA availability of the codon in the A-site.
Ponnala et al.\ use a deterministic model to represent this aspect of translation.  For each codon, they define a number
of ``wait cycles," a function of the rarity of the associated tRNA.  The number of wait cycles corresponds to the number
of times the force is allowed to increment displacement.

\section{Applications}

\subsection{Bovine Growth Hormone}
Besides the frameshifting sequence \prfB, our model also shows a significant correlation in data with biological
bovine growth hormone data, sequenced previously in \ecoli~\cite{schoner:bgh}. For example, our model purports a relation
between the deviation from the zero reading frame---for non-frameshifters anyway---and protein yield, ultimately postulating
a relation between ribosomal efficiency and its ability to maintain its reading frame without wobbling. Schoner et al.
modified beginning codons of an initial bovine growth hormone sequence to produce a series of alternate bGH sequences,
two of which produced significantly higher yield than their compatriots. Those two, dubbed pcZ101 and pcZ105, produced significantly
lower deviation than pcZ104, pcZ108, and other bGH sequences, thus agreeing with the biological data from Schoner et al.
A model's accurate prediction of yield given a sequence would greatly benefit the work of biologists across the world,
aiding in verifying the yields of a large sequence of proteins especially when experiments on all of them proves too difficult
or costly. Therefore, bGH is an important cornerstone of our computational model's impact on biology.

\subsection{rpoS}

\bibliography{wizards}
\end{document}