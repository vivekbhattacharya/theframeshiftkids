\documentclass[12pt, draft]{article}
\usepackage{candor, setspace}
\linespread 2
\numberwithin{equation}{section}

\begin{document}
\section{Introduction}

\section{Description of the Model}
\subsection{Free Energy}
\label{freeenergy}

Free energy in a cell arises from the hybridization between two
sequences of RNA and drives ribosomal translation~\cite{starmer}.
\citet{weiss88} suggests the 16S tail of the ribosome hybridizes with its mRNA strand;
from this, we can calculate the free energy values for the interaction between the two strands of RNA.
\citet{freier}, in turn, proposes a thermodynamic model for exactly this,
modeling the interaction in terms of \emph{doublets}, i.e. pairs of consecutive nucleotides.
Using this model, Freier numerically calculated the free energy available
for the hybridization of all RNA doublets permutations.

% Hao: I took out the setting energy values due to irrelevance.
From this, we can simulate hybridization between the associated tRNA, correspond to the number
of times the force can increment displacement from the current reading frame.
In their deterministic model, the force acts for \emph{exactly} this set number of cycles given a codon.

\subsection{Frameshifts}

% Where the following concept should be first mentioned, I'm not sure -- Vivek

First, we let a displacement of $x = 0$ correspond to the zero reading frame and increments of
\emph{two} to represent a one-nucleotide change. For example, $x =2$ represents a +1 frameshift.
Ponnala~\cite{lalit:embs} prove that both $x = 0$ and $x = 2$ are fixed (stable) points in displacement in his model.

A sudden jump from approximately $x = 0$ to $x = 2$ is then the first indication of a $+1$ frameshift.
In essence, this jump suggests that the ribosome jumps one entire base pair in the mRNA sequence.
Figure~\ref{prfB:lalitdisp} shows the displacement plot per this deterministic model for \prfB, 
a gene known to have a programmed $+1$ frameshift at codon 25.

In conjunction with this characteristic plot, a $+1$ frameshift also shows a 240\degree
rotation of phase angle from the species angle, consistent with the creation of displacement data from the phasors.
Intuitively, a frameshift is only sustained if the free energy signal aligns itself with the sudden jump in displacement.
Since the free energy signal has a period of one codon~\cite{lalit:mechanics}, for a $+1$ frameshift the free energy signal
must undergo a phase shift of one-third of an entire period (Figure~\ref{prfb:polar}).

\subsection{A Stochastic Model for Displacement}

% Find an example of a gene with a screwed-up displacement plot under Lalit's old model. --Vivek

The gene \prfB\ exhibits a characteristic frameshift under the deterministic model: The displacement jumps to $x=2$.
Certain genes, however, demonstrate ambiguous behavior near $x = \pm 1$.
In a deterministic model, the fixed displacement plots implies
we cannot clearly interpret whether this behavior indicates a tendency for a stable frameshift.
In practice, ribosomal translations are not deterministic. Due to the presence of
noise within the cell environment, we instead chose to model translation stochastically, adding
randomness to the choice of reading frame a ribosome executes at each codon given the free energy signal.
As such, we introduce probability into the model.

We propose that at each cycle in elongation, the ribosome must make a decision: stay in the current reading frame,
move to the $+1$ reading frame,
or move to the $-1$ reading frame.  We can further subdivide this choice into the individual wait-cycles.
At each wait cycle, the ribosome chooses from the above possibilities or proceeds to another wait-cycle without making a decision.

Let $abcd$ be a sequence of four nucleotides, with $abc$ in the
current reading frame and $bcd$ in a +1 reading frame.  Let $x$ be the
displacement of the current wait cycle of the ribosome.  As the
incremental displacement approaches +1, the probability of choosing
codon $bcd$ should increase and the probability of choosing codon
$abc$ should decrease.  We modeled this behavior using even powers of
cosine and sine functions for $abc$ and $bcd$.  We
define $\omega$ as the weight that is directly proportional to
the probability.
\begin{equation}
  \omega_{abc} = \cos^{10}{\frac{x\pi}{4}} \text{ and } \omega_{bcd} = \sin^{10}{\frac{x\pi}{4}}.
\end{equation}
For the purposes of this model, the cosine and sine functions are taken to the tenth power, but
in future studies, this parameter can be modified as long as it stays an even positive integer.
The functions have a period of two base pairs ($x=4$), which is biologically sound:
If the ribosome lies completely in the 0 frame, then the probability of staying in the 0 frame should be 1.  
Consequently, if the ribosome lies in the +1 frame ($x=2$), the probability of going to
the 0 frame should be 0.

Now we define a value $n_{abc}$ based on normalizing $N_{abc}$, the
number of wait cycles, itself inversely proportional to the
tRNA availability of codon $abc$.
We want to wait long enough that the probability of assume the probability of frameshifting after
after $N_{abc}$ wait cycles is $\frac{1}{2}$.  Let $P$ be the
probability of moving on to the next codon and staying in current reading
frame.  Then we should have $1-\left(1-P\right)^{N_{abc}} =
\frac{1}{2}$.  Because $P \propto \omega_{abc}$, we define $n_{abc}$ to be the constant of
proportionality. That is, $n_{abc} = \omega_{abc} / P$.  Coupled
with that $\omega_{abc} \le 1$, we have $n \le \sqrt[N]{2}/(\sqrt[N]{2} - 1).$
where $n = n_{abc}$ and $N = N_{abc}$. Mathematically, the probability of choosing a codon is then
\begin{align}
  \prod_{i=1}^N \left(1-\frac{\omega_i}{n}\right) \text{ where } \omega_i = \cos^{10}{\frac{x_i\pi}{4}}.
\end{align}
Here, $n$ represents the wait time for that codon and $N$ is the ordinal of the wait cycle. (For example,
$N=1$ on the first wait cycle, $N=2$ on the second, and so forth.)
It is important to note that $\omega$ is dependent on displacement, 
which increments after every wait cycle.

\section{Studying \prfB}

\subsection{Results from the Stochastic Model}

In \ecoli, the gene \prfB\ codes for protein release factor B, an essential element in translation.
This gene, as mentioned, is known to have a programmed frameshift at
the 25$^{\textrm{th}}$ codon.
Our proposed stochastic model, as with the original model, can accurately predict this frameshift.
Figure~\ref{prfb:disp} shows a displacement plot for \prfB, again with a distinctive jump at codon 25.
To account for random variation, ten different runs are shown on the same set of axes.
\footnote{Note that the polar plot is the the same as Figure~\ref{prfB:polar}; 
the new model does not alter the polar plot and the free energy calculations.}

Notably, the displacement plot does not reach $x=2$ over the span of one codon, as the old model predicted.
Rather, due to randomness, the ribosome chooses the codon in the $+1$ frame before actually reaching a displacement of exactly 2.
The propensity to approach $x=2$, however, concurs with the biological idea that the ribosome stays in frame after the frameshift.

\subsection{Biological Evidence}

\citet{weiss87,weiss88} conducted research on the
frameshift in \prfB, determining factors that would affect
translational rate.  We repeated Weiss's experiments computationally
using the stochastic model, which agreed with Weiss's results and
provides strong evidence that our model accurately represents existing
biological evidence and metrics.

% We need plots.

Specifically, \citeauthor{weiss88} moved the stop codon in the
\prfB\ sequence one nucleotide upstream, causing the sequence to fail to
frameshift. Moving the stop codon another nucleotide upstream failed
to create a frameshift as well. However, a final nucleotide upstream
transposition resulted in a similar frameshift to the initial \prfB sequence.

% More plots needed.

% Another aspect of \citeauthor{weiss88}'s  work involved changing the 16S tail of the ribosomal RNA.

\section{Applications: Bovine Growth Hormone}
Today, the field of genetics has widely intersected with
agriculture where synthesized compounds keep the machinery of
production running day in and day out. For example, recent
research~\cite{schoner:bgh} have attempted to amplify the DNA sequence
that codes for bovine growth hormone in cows and attempt to produce it
in \ecoli. However, the process for modifying a DNA sequence is slow
and arduous in a biological experiment environment especially in the
measure of protein yield for each newly modified sequence. However,
with the help of our computation model, we can easily determine the
relative yields for each sequence quickly and efficiently if such a
correlation between the metrics our model outputs and biological
research exists. Indeed it does.

Schoner~\cite{schoner:bgh}, primarily modifying the initial codons of an initial
bovine growth hormone sequence, found two sequences pcZ101 and pcZ105,
to have particularly high protein yield and efficient with respect to
the six other sequences. Our model agrees. In modeling displacement,
we found pcZ101 and pcZ105 to have the least reading frame deviation
from $x = 0$ (Table~\ref{bgh:deviation}), which in our model~\cite{lalit:mechanics} indicates
higher yield as increasing deviation from the correct reading frame
produces larger error within the ribosome during translation.

\begin{table}[tbp]
\begin{center}
    \begin{tabular}{lc}
        \toprule
        \textbf{Sequence} & $\mathbf{\sigma}$\\
        \midrule
        pcZ101 & 0.76719\\
        pcZ105 & 0.76710\\
        \midrule
        pcZ100 & 0.80020\\
        pcZ104 & 0.79729\\
        pcZ108 & 0.79241\\
        pcZ110 & 0.79760\\
        pcZ112 & 0.79419\\
        pcZ115 & 0.79794\\
        \bottomrule
    \end{tabular}
    \caption{Deviation for bGH Sequences with Sample Size~200}
    \label{bgh:deviation}
\end{center}
\end{table}

In addition, in examining the polar plots, we noticed that during
early translation the phase of translation for pcZ101 and 105 tended toward the species
angle -30 degrees before diverging toward approximately 15 degrees. As
it stands, our model, as with displacement, indicates higher yield when the polar plot
indicates less deviation from the correct (zero) reading frame, as it
does here with the species angle~\cite{lalit:mechanics}. Therefore our
model is consistent with research~\cite{bgh:initiation} that indicates
the early codons have a dramatic impact on the ultimate efficiency of
ribosomal translation.

These results from bovine growth hormone that correlation with a broad
spectrum of known ribosomal behavior indicate our computational model
can significantly increase the speed at which geneticists and
biologists can obtain valuable information when synthesizing
commercially or medicinally useful compounds without laboriously
working through biological experiments.

\subsection{Calibrating tRNA Availabilities}
Although our model relies upon tRNA availabilities in the calculation of
wait time from free energy signals, we lack a reliable source for such
values. However, with the model, we can calibrate such values by
surveying existing, measured \ecoli\ genes. For example, two bovine growth
hormone proteins (pcZ101 and pcZ105)  have much higher expression
levels than their siblings as discussed below. To this end, we
developed a genetic algorithm attempting to find the TAV values that
produces the optimal separation between pcZ101 and pcZ105 yields in
comparison to other bGH proteins.

First, we generate a list of randomly modified TAV vectors at which
point we calculate the ratio of the deviation from
displacement---discussed above---for pcZ101 and pcZ105 to the other
six bGH proteins. From there, we sort the modified TAV according to
this ratio and discarding the least optimal half of the tRNA
vectors. We then choose two of the remaining vectors, with the chance
of each vector being proportional to its rank.  We then take a
weighted average of the two vectors and spawn a new vector.  After
creating the next generation according to a constant ``gene pool''
size, we delete the previous generation and repeat. After a fixed
number of generations, the algorithm terminates and returns the
optimal TAV vector.

\section{Applications: rpoS}
% I'm colluding deviation and probability yield here, but it's for a
% good purpose. --Hao.

\citet{rpos:process} indicate that rpoS, a gene that codes for an RNA
polymerase sigma factor, contains sections of rare codons that disrupt
ribosomal translation. Indeed, our computation model agrees with this
biological evidence, showing again moderately high deviation from $x =
0$. In mass production of such a polymerase sigma factor, biologists
can replace sequences of codons known to add noise and error to
ribosomal translation with synonymous codons (see above
footnote). However, with a computation model, the process is much
faster and, with this performance, we can perform a randomized, greedy
algorithmic search for codon sequence replacements. We first find an
early trouble spot of four codons, randomly replacing that sequence
with synonymous four codons, and running our model against those
permutations to obtain a locally optimal sequence at that place. We
then repeat for all trouble spots\footnote{Specifically, places with
  mistake frameshifts in our model, which correlate to rare codons per
  above discussion} the first, terminating when we have locally
optimized the last one. With this algorithm, we reduced our standard
deviation metric for the rpoS displacement plot from 0.168 to 0.117
on sample size of 1000 with a replacement of 33 codons. The
initial correlation between deviation and biological expression
provides strong anecdotal evidence that, with future biological
experimentation, our algorithm has indeed increased protein yield
bypassing a slow and arduous biological experimentation process.

\section{Verification}
\begin{figure}
  \caption{Artificial linker sequence}
  \label{linker}
  \begin{verbatim}
    agaaaucagacc
    aug gag gcu ggc acc agg ggg uac agu uaag caa acg uag
  \end{verbatim}
\end{figure}

In order to verify the model's ability to predict frameshifts, we
artificially created a thirteen-codon sequence (Figure~\ref{linker})
with a twelve-acid leader that frameshifts at the ninth codon and,
from a BLAST search, is not known to nature.  Using a fused protein of
beta-galactosidase, our linker, and xylE, we measured levels of our
synthetic compound. $\beta$-gal and xylE constructs retain their
functionality and thus we can measure protein expression during an in
vivo experiment. For example, beta-galactosidase levels correlate to
levels of nitrophenol, which has a yellow color, while xylE expresses
an enzyme that cleaves catechol, another metric that correlates to the
expression of our linker sequence.

% Talk about plots. --Hao

\begin{singlespace}
  \bibliography{wizards}
\end{singlespace}
\end{document}
