\documentclass[twocolumn, article, oneside]{memoir}
\usepackage{candor, verbatim, abstract}
\usepackage[charter]{mathdesign}
\renewcommand{\abstractname}{}
% Suppress memoir's fancy headings.
\pagestyle{plain}
\linespread{1.2}

\hyphenation{frame-shifts gen-bank da-ta-base hy-bri-di-za-tion}

\newcommand{\BWFtitle}[1]{A Computational Model for Translational
  Efficiency and Frameshifts in #1{Escherichia coli} Using a Genetic Signal
  Processing Approach}
\newcommand{\BWFauthors}{Hao Lian, Vivek Bhattacharya, and Daniel
  R. Vitek}

\usepackage[final, colorlinks=true, linkcolor=BWFBlue,
  citecolor=BWFGreen, urlcolor=BWFRed, pdftitle={\BWFtitle{}},
  pdfauthor={\BWFauthors}, pdfcreator={The Frameshift Kids},
  pdfstartview={FitH}]{hyperref}

\author{\BWFauthors}
\title{\BWFtitle{\emph}}
\renewcommand{\maketitle}{%
  \Large \textbf{\thetitle} \par \theauthor \par
}

\begin{document}
\section{Question}
To what extent can we successfully model ribosomal translation within
\ecoli\ computationally and how can that model help predict
translational efficiency and frameshifts, both programmed and
accidental?

\section{Hypothesis, Problem, Goals}

\section{Procedures}
We used Matlab and Perl to extend a previous computational model. We
collected all the data through the Matlab graphical user interface,
which represented the output from the Matlab and Perl programs.

\subsection{Analysis}

\phantomsection
\addcontentsline{toc}{section}{Bibliography}
\bibliography{wizards}
\end{document}

% This is for ispell. Do not delete. --Hao
% LocalWords:  abcd abc bcd sp pcZ riboproteins disp aag guu efr ikemura jbsb
% LocalWords:  kane sd xray starmer aa TAV uga NCBI's Genbank lacZ hui UAA TAVs
% LocalWords:  Ecogene schoner kozak
