\documentclass[12pt,titlepage]{article}

\usepackage{times}
\usepackage{fullpage}
\usepackage{amsmath}
\usepackage{vector}

%%%%%%%%%%%%%%%%%%%%%%%%%%%%%%%%%%%%%%%%%%%%%%%%%%%%%%%%%%%%%%%%%%
% A note to all for now.  I am just putting down ideas on paper. %
% The citations are done very informally, and we will definitely %
% have to look up more stuff to cite some of the ideas I mention %
% in this paper.  Daniel, this is also mainly your job!!         %
%%%%%%%%%%%%%%%%%%%%%%%%%%%%%%%%%%%%%%%%%%%%%%%%%%%%%%%%%%%%%%%%%%

\providecommand{\hj}{\textbf{(cite)}}
\numberwithin{equation}{section}

\linespread{2}

\begin{document}

\section{Introduction}
\label{sec:intro}

Recombinant DNA technology holds much promise as a means of creating proteins in large quantities for a variety of reasons---ranging from artificial proteins, like insulin \hj, to green enzymes for paper bleaching, for example. Yet, this technology has yet to revolutionize society in a manner that many have often fantasized.  The prime explanation for this unfortunate situation implicates the idea of cross-species translation.  For various biological reasons, genes that express highly in certain species, like \emph{Homo sapiens}, for example, often are translated poorly by ``lab rats'' like \emph{Escherichia coli} \hj.  For precisely this reason, many years of trial-and-error are necessary to modify a gene sequence enough for it to be produced in considerable quantities in \emph{E. coli} and other such bacteria.

In this paper, we propose a theoretical model to predict translational efficiency of a given gene in a given organism.  We also provide an algorithm that automatically modifies a given gene sequence to improve translational efficiency.  We believe that this algorithm provides a solid basis for furthering the use of recombinant DNA technology in numerous sections of modern society.

\subsection{Protein Translation}
\label{sec:translation}

Protein translation refers to the latter step in this process: converting mRNA to proteins.  Messenger RNA carries the message encoded in DNA to the ribosomes in either the cytoplasm or the rough endoplasmic reticulum, where the process actually occurs.

In a very simplified model, the 30S and 50S subunits of the ribosome bind together such that the start codon (AUG) of the mRNA sequence lies at the P-site.  Transfer RNA's bring amino acids to the ribosome, specifically matching the codon in the A-site.  Once an amino acid does arrive a the ribosome, a peptide bond forms between the new amino acid and the existing polypeptide attached to the tRNA at the P-site.  Through a series of mechanisms, including GTP hydrolysis and use of Tu elongation factor, the mRNA moves by one codon along the ribosome.  The tRNA in the P-site enters the E-site and is ejected from the ribosome, transferring the polypeptide to the new tRNA at the P-site.

Since the process of protein translation occurs at a submicroscopic level, much of the details are still unclear.  However, a strong amount of research does suggest that the 3' tail of the 16S subunit of the ribosome plays a distinct role in translation.  Specifically, Weiss et al.\ have shown that the 16S tail is involved in the frameshift of the {\emph{prfB}} gene \hj.  Scientists have also hypothesized that RNA hybridization plays a role in the interaction of the tail with the messenger \hj.

\subsection{Free Energy}
\label{sec:freeenergy}

Much research has been done on methods of calculating free energy upon Watson-Crick hybridization of RNA strands \hj.  Then nearest neighbor (NN) model, described in Freier et al.\ \hj, is probably the most popular such model.  It calculates energy based on the idea that the stability of a certain base pair depends on which base pairs neighbor it.  The total free energy is a sum of the helix initiation energy of the first base pair, propogation energies for subsequent base pairs, and a correction for self-complimentary sequences \hj.  Various scientists have calculated values for free energy values in base pairing \hj.

The idea of free energy and complimentarity bears special significance to translation because of the potential role of the 16S tail.  Research has shown that the tail contains sections very similar to certain sequences in the mRNA of \emph{E. coli}.  Specifically, the Shine-Dalgarno region has been of interest because of its role in initiating translation \hj. (<== Add to this, but I'm not sure how.)

Work by Ponnala et al.\ \hj has shown that the free energy signals as the tail moves down the mRNA show a distinct sinusoidal pattern with a period of one codon ($f = 1/3$).
\end{document}