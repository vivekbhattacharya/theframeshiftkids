\documentclass[10pt,titlepage,twocolumn, draft]{article}
\usepackage{fullpage, amsmath, amssymb}
\bibliographystyle{abbrv}

\begin{document}
\section*{Hello}
Vivek, Hao, and Daniel authored this report on August To Be Determined, 2007.

\section{The old}
The old displacement model~\cite{lalit:mechanics}~\cite{lalit:embs} was deterministic
in which the change in displacement $\Delta x$ arose from the instantaneous energy available at
each codon with codon choice limited to the -1, +1, and 0 reading frames, depending on
which interval among $(-\infty, -1]$, $(-1,1)$, and $[1, \infty)$ contained $x$.
This established a recursive relationship between instantaneous energy,
which relies on the wait time for a codon, and codon choice.

\section{The new}
The new model is stochastic, arising from talks with Dr. Bitzer about sinusoidal probability.
Let the current frame be $f$. The probability of choosing
$f-1$, remaining at $f$, or choosing $f+1$ depend on sinusoidal piecewise functions defined
on intervals that correspond to hugs, such as the function
    \begin{align}
        q(x) &=  \cos \frac{x \pi}{T}  & \text{if $-\frac{T}{2} < x < \frac{T}{2}$}\\
        q(x) &=  0 & \text{otherwise}
    \end{align}
that describes the probability the model will remain at the $f$ frame. (Under normal conditions
when no frameshift occurs, $f = 0$.)

\bibliography{wizards}
\end{document}