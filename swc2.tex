\documentclass[12pt, draft]{article}
\usepackage{candor, setspace}
\usepackage[verbose]{wrapfig}
\newcommand{\BWFtitle}{Computationally
  Modeling Ribosomal Translation and Programmed Frameshifts in}
\newcommand{\BWFauthors}{Hao Lian, Vivek Bhattacharya, and Daniel
  R. Vitek}

\usepackage[final, colorlinks=true, linkcolor=BWFBlue,
  citecolor=BWFGreen, urlcolor=BWFRed, pdftitle={\BWFtitle Escherichia
    coli}, pdfauthor={\BWFauthors},
  pdfsubject={Bioinformatics, Genetics, and Biology}, pdfcreator={The
    Frameshift Kids}, pdfkeywords={bioinformatics,genetics,biology,
    frameshifts,perl,matlab,ecoli,prfb,rpos,bgh}, pdfstartview={FitH},
  backref]{hyperref}

\linespread 2
\numberwithin{equation}{section}

\author{\sc{\BWFauthors}}
\date{{\sc \today}}
\title{\bf{\BWFtitle~\emph{Escherichia coli}}}

\begin{document}
% Title, table of contents, and abstract start things off.
\pagenumbering{roman}
\begin{singlespace}
  \maketitle
  \tableofcontents
\end{singlespace}

\clearpage
\begin{abstract}\begin{normalsize}
  In this document, we discuss a stochastic model for ribosomal displacement relative to reading 
  frame based on forces arising from changes in free energy present in hybridization between the 
  16S rRNA tail and nucleotides upstream from the A-site.  We present both our investigation and 
  potential applications for this model, including optimization of economically significant gene
  sequences such as bovine growth hormone, non-experimental prediction of translational efficiency for man-made 
  proteins, adjustment of calculated tRNA availability values, and modelling of frameshifts in \prfB\ in \ecoli.
  We present this investigation as a series of various inquiries made 
  in subjects ranging throughout the field of molecular biology.  We also applied our stochastic 
  model to create a short, 16-codon sequence that is not known to produce a frameshift in nature 
  in order to test our model's predictive power for synthesized mRNA sequences, and designed an 
  experiment to verify this frameshift.  We expect results in October 2007.
\end{normalsize}\end{abstract}
\clearpage
\pagenumbering{arabic}

% Insert introduction here.
\section{Computational Methods}
\subsection{Free Energy}
\label{freeenergy}

Free energy in a cell arises from the hybridization between two
sequences of RNA and drives ribosomal translation~\cite{starmer}.
\citet{weiss88} suggest the 16S tail of the ribosome hybridizes with its mRNA strand;
from this, we can calculate the free energy values for the interaction between the two strands of RNA.
\citet{freier} propose a thermodynamic model for exactly this,
modeling the interaction in terms of \emph{doublets}, which are pairs of consecutive nucleotides.
Using this model, Freier calculated the free energy available
for the hybridization of permutations of RNA doublets.

\subsection{The Deterministic Model}
\citet{lalit:mechanics} assume a sinusoidal model for
free energy, whence we can project free energy onto magnitude and
phase through a memory model. Ponnala simulates free
energy in a memory structure that can store three values (registers)
that he later can represent as a phasor, a concept from physics. Then,
we can visualize free energy from polar plots and deduce frameshifts,
which occur along defined boundaries~\citet{lalit:mechanics} on the polar plot for a species.
 
\citet{lalit:embs} then represent the cumulative phasor
at codon $k$ as $\bvec{V} = Me^{i\theta}$ where $i$ is the imaginary
constant, allowing us to calculate the magnitude and phase at codon
$k$ through simple trigonometry. Magnitude and phase are then modeled
as a polar plot, one of the model outputs. Differentiating this vector, we
arrive at vector $\bvec{D}$ for instantaneous energy, interpreting
this as a force that acts on the ribosome, keeping the mRNA in
a given reading frame.
 
The length of time that the force acts on the ribosome depends upon
the codon's tRNA availability at the A-site during translation.
\citeauthor{lalit:mechanics} use a deterministic model to represent this: For each codon,
a number of ``wait cycles," dependent upon the rarity of the
associated tRNA, correspond to the number of times the force can
increment displacement from the current reading frame.  In the
deterministic model, the force acts for \emph{exactly} this set number
of cycles given a codon. We can then simulate hybridization between the
16S ribosomal subunit and a given mRNA strand: First, the 13-base 16S
tail of \ecoli\ hybridizes with the first 13 bases of a sequence,
which is a 12-base leader sequence and the first base of the start
codon \textsc{aug}, to determine the free energy value of the first acid.
Then, shifting one base pair at a time, we calculate the free energy
for the entire sequence according to \citet{starmer}.

\subsection{Frameshifts}
\label{section:frameshifts}

\begin{cfigure}
  \caption{Deterministic displacement plot of~\prfB}
  \label{prfB:deterministic}
  \includegraphics[scale=0.4]{prfB/deterministic}
\end{cfigure}

\begin{wrapfigure}{R}{0.5\textwidth}
  \caption{Polar plot of \prfB}
  \label{prfB:polar}
  \includegraphics[width=0.5\textwidth]{prfB/polar}
\end{wrapfigure}

First, we let a displacement of $x = 0$ correspond to the zero reading
frame and increments of \emph{two} to represent a one-nucleotide
change. For example, $x =2$ represents the +1 frame.
\citet{lalit:embs} prove that both $x = 0$ and $x = 2$ are fixed
(stable) points in displacement in their model, as expected.

A sudden jump from approximately $x = 0$ to $x = 2$ is then the first
indication of a $+1$ frameshift; it suggests the ribosome skips one
entire base pair in the mRNA sequence.  \autoref{prfB:deterministic}
shows the displacement plot per this deterministic model for \prfB, a
unique gene with a programmed $+1$ frameshift at codon 25. In
conjunction with this characteristic plot, a $+1$ frameshift also
displays an equally characteristic clockwise 120\degree\ phase angle
rotation from the species angle, an emergent property from
constructing displacement vector from the phasors.  Intuitively, a
frameshift sustains if the free energy signal aligns with the sudden
jump in displacement.  Since the free energy signal has a period of
one codon~\cite{lalit:mechanics}, a $+1$ frameshift the free energy
signal must undergo a phase shift of one-third of an entire period
(\autoref{prfB:polar}).

\subsection{A Stochastic Model for Displacement}
\label{stochastic}

As mentioned, the gene \prfB\ exhibits a programmed frameshift under
the deterministic model: The displacement jumps to $x=2$.  Certain
other genes, however, demonstrate equivocal, ambiguous behavior near
$x = \pm 1$~\cite{lalit:embs}.  In a deterministic model, we lack the
information and sensitivity needed to discern clearly a tendency for a
stable frameshift. Even worse, the model may not show these unstable
behaviors at all. In practice then, ribosomal translations are not
deterministic. Due to the presence of noise within the cell
environment, any deterministic model ultimately imperfectly accounts
for translation. Therefore, we next explored stochastically modeling
ribosome translation with a sinusoidal probability paradigm that
follows.

We propose that at each cycle in elongation, the ribosome must make a
decision: stay in the current reading frame, move to the $+1$ reading
frame, or move to the $-1$ reading frame.  We can further subdivide
this choice into the individual wait-cycles.  At each wait cycle, the
ribosome chooses from the above possibilities or proceeds to another
wait-cycle without making a decision.  In addition, we assert the 
number of wait cycles is inversely proportional to the tRNA availability of 
the codon in the current reading frame, as rarer codons should force the 
ribosome to wait longer for a correct transfer. In turn, we calculate
tRNA availability from existing research suggesting an association
between it and codon frequency~\cite{ikemura}.

Let $abcd$ be a sequence of four nucleotides, with $abc$ in the
current reading frame and $bcd$ in a +1 reading frame.  Let $x$ be the
displacement of the current wait cycle of the ribosome.  As the
incremental displacement approaches +1, the probability of choosing
codon $bcd$ should increase and the probability of choosing codon
$abc$ should decrease.  We model this behavior using even powers of
cosine and sine functions for $abc$ and $bcd$.  We
define $\omega$ as the \emph{weight} that is directly proportional to
the probability.
\begin{equation}
  \omega_{abc} = \cos^{10}{\frac{x\pi}{4}} \text{ and } \omega_{bcd} =
  \sin^{10}{\frac{x\pi}{4}}.
  \footnote{For the purposes of this model, the cosine and sine
    functions are taken to the tenth power, but in future studies,
    this parameter, which must be an even integer, can change.}
\end{equation}
If the ribosome lies completely in the 0 frame, then the probability
of staying in the 0 frame should be 1.  Consequently, if the ribosome
lies fully in the +1 frame ($x=2$), the probability of going to the 0
frame should be 0. Therefore, these functions have a period of two
base pairs ($x=4$).

We define $N_{abc}$ to be the number of wait cycles 
allocated to the given codon in the current frame.
Suppose we are at a particular wait cycle on codon $abc$ and
abbreviate $N = N_{abc}$ as per above.
In absence of further research, we assume the probability
of frameshifting is 1/2.  Let $P$ be the instantaneous probability of
moving on to the next codon and staying in the current reading frame.
Then we should have $1 - \left(1-P\right)^{N} = \frac{1}{2}$.
Given that $P \propto \omega$, let $n = n_{abc}$ be the constant of
proportionality; thus $n = \omega / P$.  We have
$\omega \le 1$, so hence $n \le \sqrt[N]{2}/(\sqrt[N]{2} - 1)$
Mathematically, the
probability of choosing the codon at a given wait cycle in the zero reading frame is just

\begin{equation}
  \prod_{i=1}^K \left(1-\frac{\omega_i}{n}\right) \text{ where }
  \omega_i = \cos^{10}{\frac{x_i\pi}{4}}.\footnote{The sinusoidal
    component of $\omega_i$ is not always cosine and depends upon the
    frame choice in question.}
\end{equation}

Here, $n$ represents the wait time for that codon and $K$ is the
ordinal of the current wait cycle. For example, $K=1$ on the first
wait cycle, $K=2$ on the second, and so forth.  Importantly, $\omega$
is dependent on displacement, which increments after every wait cycle.

Importantly, we have $\displaystyle\lim_{N\rightarrow\infty} n = N/\ln{2}$, indicating 
that the probability of choosing a codon at a given cycle is directly proportional 
to its tRNA availability, another result that coincides with biological evidence.
\section{Analysis}
In order to test our computational model, we ran a number of
experiments to analyze sequences present in the \ecoli\ genome.

\subsection{Measures}
\label{section:metrics}

As this model is stochastic, multiple runs (the sample size) of the same sequence must be analyzed.
As such, we propose two metrics for analysis of a number of different runs 
simultaneously. 

\subsubsection{Error-Free Rate}
When studying a
sequence with a programmed frameshift, it measures the percentage of runs 
during which the ribosome chooses the correct codon
at \emph{every} juncture.  For a +1 frameshift, for example, the ribosome must
choose the +1 frame at the frameshift codon and stay in the 0 frame before
and the +1 frame after in order for the run to be a success.

\subsubsection{Displacement Deviation}
\label{section:deviation}

We propose the formula
\begin{equation}
    d = \sqrt{\frac{\sum_i \left(x_i - \beta_i\right)^2}{N}}
\end{equation}
where $\beta_i$ is the predicted reading frame at codon $i$, $x$ is
the displacement at codon $i$, and $N$ is the total number of codons
as a measure of the deviation of the sequence from the expected
reading frame.  Usually $\beta_i = 0$ unless a programmed frameshift
exists as it does for \prfB.  For example, for \prfB\ $\beta_i = 2$
for all $i \geq 25$ because \prfB\ frameshifts at codon 25
\textsc{uga} and the model represents a frameshift with +2
displacement per \autoref{section:frameshifts}.

\subsection{\prfB\ and Related Sequences}
As discussed, \prfB\ codes for protein release factor B in \ecoli.
Notably, microbiologists agree that \prfB\ has a programmed frameshift
in the 25$^\textrm{th}$ codon~\cite{weiss87}.  We thus test the
gene \prfB\ to determine whether our model can predict this
frameshift.  We downloaded the nucleotide sequence for \prfB\ from
NCBI's Genbank database.  [Dr. Stomp, do we need a citation for Genbank?]

\citet{weiss87} performed a number of biological experiments on
\prfB\ to test how mutations in the sequence would affect the rate of
frameshifting.  They present a total of 35 sequences in their paper,
along with relative measures of translational efficiency.  We ran all
these through our model and correlated them with their error-free rates.
We hypothesize that genes found to frameshift at high rates by
\citeauthor{weiss87} should also show high error-free rates under
our model.

\subsection{The \ecoli\ Genome and Ribosomal Proteins}
We now aim to test the capabilities of our model to predict translational
efficiency other than frameshift rate.  We hypothesize that displacement
deviation provides a suitable metric for translational efficiency:  A lower
deviation would correspond to a more efficient sequence.  

We base this 
hypothesis on the fact that as the displacement moves away from the 
horizontal axis, the ribosome has a greater probability of picking the
codon in the +1 frame by chance, in effect producing junk.  In addition,
a greater deviation from the zero axis inherently implies a longer
time for translation, since the force must act on the ribosome for an 
extended period of time to make the displacement relatively large.  A
high value would thus indicate a propensity for a high time for translation,
thus reducing translational efficiency.  [Better explanation, citations.]

To this end, we start with a simple fact:
Microbiologists agree that translation in \ecoli\ is an efficient process;
most regulation is transcriptional.  As such, the majority of the genes 
in the genome should exhibit low displacement deviations.  Moreover,
ribosomal proteins are known to be abundant in the cell, and thus
are hypothesized to be translated at especially efficient rates.  We
furthermore predict that ribosomal proteins will have deviations lower
on average than those for the other \ecoli\ genes.

\subsection{Bovine Growth Hormone}
We finally test our model on a set of sequences known to be translationally 
regulated.  \citet{schoner:bgh} developed a number of sequences that
code for [Guys, finish this].

\section{Parameters}
\label{section:parameters}
For these plots, we used a species angle of $\theta_{\rm{sp}}
=-30\degree$.

\section{Results}
\subsection{\prfB}

\begin{cfigure}
  \caption{Stochastic displacement plot of \prfB}
  \label{prfB}
  \includegraphics[scale=0.4]{prfB/disp}
\end{cfigure}

In \ecoli, the gene \prfB\ codes for protein release factor B, an
essential element in translation.  This gene, as mentioned, is known
to have a programmed frameshift at the 25$^{\textrm{th}}$ codon.
\autoref{prfB} shows a displacement plot for
\prfB, again with a distinctive jump at codon 25.
\footnote{Note that the polar plot is the the same as \autoref{prfB:polar}. 
The new model does not alter the polar plot or the free energy calculations.}

Notably, the displacement plot does not reach $x=2$ over the span of
one codon, as the old model predicted.  Rather, due to randomness, the
ribosome chooses the codon in the $+1$ frame before actually reaching
a displacement of exactly 2.  The propensity to approach $x=2$,
however, concurs with biological evidence indicating the ribosome
stays in frame after the frameshift.  [Dr. Stomp, we need references
  here.]

\begin{cfigure}
  \caption{Sensitivity plot for \prfB}
  \label{prfB:sens}
  \includegraphics[scale=0.4]{prfB/sensitivity}
\end{cfigure}

We repeated Weiss's experiments computationally
using the stochastic model, and our results agreed~\cite{weiss87,weiss88}.
This concurrence again provides support for the biological validity of
the stochastic model.

Specifically, \citeauthor{weiss87} moved the stop codon in the
\prfB\ sequence one nucleotide upstream, causing the sequence to fail to
frameshift. Moving the stop codon another nucleotide upstream failed
to create a frameshift as well in our model.

\subsection{\ecoli\ genes}
\begin{cfigure}
  \caption{Displacement deviation for \ecoli\ genes with with
    deviation >3 (<1\%) truncated}
  \label{ecoli:hist}
  \includegraphics[scale=0.4]{histograms/everything}
\end{cfigure}

\ecoli, as a product of evolution, is naturally efficient in its
processes. [Dr. Stomp, we need references here.]  It logically follows
that the model should indicate low deviation for most
\ecoli\ proteins.  \autoref{ecoli:hist} is a histogram of the
deviation yields of 4364 genes of \ecoli, over 80\% of the entire
genome.  As predicted, 93.45\% of the genes that we ran laid in the 0
to 1 interval, agreeing with our theory of natural efficiency.  The
average displacement deviation for these \ecoli\ genes is 0.4425, in a
run for 500 iterations per gene.

\begin{cfigure}
  \caption{Boxplot comparison of ribosomal proteins and an almost
    complete sample of verified \ecoli\ genes}
  \label{ribosomal:comp}
  \includegraphics[width=0.8\textwidth]{histograms/ribosomal}
\end{cfigure}

\subsection{Ribosomal Proteins}
\label{section:riboproteins}
To compound this finding, biologists also agree that ribosomal
proteins offer especially high translational efficiencies, since the
cell must produce them in such large quantities. [Dr. Stomp, we need
  references here.] The displacement deviation from $x=0$ for ribosomal proteins
is on average 0.2708 in comparison to the average of 0.4425 for our
large sample of \ecoli\ genes, which is significantly higher with a $t$-value of
0.0109 when performing a two-sample $t$ test on the means.

\subsection{Bovine Growth Hormone}
\label{section:bgh}

We investigated the concept of deviation yield as a measure of biological
yield by studying bovine growth hormone (bGH), a protein commonly used
in agriculture.
Research~\cite{schoner:bgh} at the time attempted to produce bGH
in \ecoli\ in large amounts.

\begin{tabular}{lccc}
  \toprule
  \textbf{Sequence} & $d$ & $\sigma(d)$ & Yield (\% bGH)\\
  \midrule
  pCZ101 & 0.5146 & 0.03313 & 30 \\
  pCZ105 & 0.5139 & 0.04181 & 34\\
  pCZ112 & 0.6612 & 0.03633 & 33\\
  pCZ115 & 0.6721 & 0.03792 & 32\\
  \midrule
  pCZ100 & 0.7107 & 0.01715 & $<$ 0.5\\
  pCZ104 & 0.7162 & 0.01433 & $<$ 0.5\\
  pCZ108 & 0.5912 & 0.05976 & 1.7\\
  pCZ110 & 0.7026 & 0.01966 & $<$ 0.5\\
  \bottomrule
\end{tabular}


\citet{schoner:bgh}, primarily modifying the initial codons of an
initial bovine growth hormone sequence, found sequences pcZ101,
pcZ105, pcZ112, and pcZ115, to have particularly high protein yield
and to be efficient in comparison to the six other sequences. In
modeling displacement, we found these four sequences  to have the least
displacement deviation from $x = 0$
(\autoref{bgh:deviation}). \autoref{bgh:disp} shows the displacement
plots of all the bGH sequences on the same set of axes. Although not
immediately obvious, the four sequences do indeed indicate lower
deviation. Despite running the genetic algorithm, pcZ108 remains an
outlier because \citeauthor{schoner:bgh} reported low protein
efficiency whereas our model reports low deviation, incongrous with
previously established correlations.

\begin{wrapfigure}{R}{0.5\textwidth}
  \caption{Displacement plot for bGH}
  \label{bgh:disp}
  \includegraphics[width=0.5\textwidth]{bgh/all}
\end{wrapfigure}

\subsection{Optimizing Translational Efficiency}

If deviation yield is in fact a potential metric for biological yield,
a method for optimizing deviation would be of use of biologists.  As such,
we wrote an algorithm to minimize deviation for given sequences.
We tested this algorithm computationally on \rpoS, a gene known to
be translationally regulated.  Biological experiments will be
conducted soon.

% I'm colluding deviation and probability yield here, but it's for a
% good purpose. --Hao.

\citet{rpoS:process} indicate that \rpoS, which codes for an RNA
polymerase sigma factor, contains sections of rare codons that disrupt
ribosomal translation. Our computation model agrees, showing
relatively high deviation from $x = 0$ in comparison to other
ribosomal proteins (\autoref{section:riboproteins}). In mass
production of such a polymerase sigma factor, biologists replace
sequences of codons with synonymous counterparts known to reduce noise
and error. However, with a computation model, the process is much
faster and with this performance we constructed a randomized and greedy
algorithmic search for such replacements. We first find an
early trouble spot\footnote{Specifically, places with mistake
  frameshifts in our model, which correlate to rare codons per above
  discussion.} of four codons, randomly replacing that sequence with
synonymous four codons, and running our model against those
permutations to obtain a locally optimal sequence at that place. We
then repeat for all trouble spots the first, terminating when we have
locally optimized the last one. With this algorithm, we reduced our
standard deviation metric for the \rpoS\ displacement plot from 0.168
to 0.117 on sample size of 1000 with a replacement of 33 codons. The
initial correlation between deviation and biological expression
provides strong anecdotal evidence that, with future biological
experimentation, our algorithm has indeed increased protein yield
bypassing a potentially slow biological experimentation process.

[Dr. Stomp, we need the notes that you mentioned.]

\subsection{An Artificial Frameshifter}
\begin{cfigure}
  \caption{Artificial linker sequence sensitivity plot}
  \label{linker:sens}
  \includegraphics[scale=0.25]{linker/sensitivity}
\end{cfigure}

\begin{wrapfigure}{R}{0.55\textwidth}
  \caption{Artificial linker sequence}
  \label{linker}
  \begin{verbatim}
    aga aau cag acc
    aug gag gcu ggc acc
    agg ggg uac agu  u  aag caa acg
  \end{verbatim}
\end{wrapfigure}

In order to verify the model's ability to predict frameshifts, we
artificially created a sixteen-codon sequence (\autoref{linker})
designed to frameshift.  The sequence is designed to frameshift into
the \textsc{aag} frame after crossing the sole uracil.  Figure~\ref{linker:sens}
shows the error-free rate of the linker sequence as a function of species
angle and initial displacement, in order to demonstrate robustness,
similar to the \prfB\ sensitivity plot (\autoref{prfB:sens}).

A BLAST search finds about thirty matches of our sequence within areas of
bacterial genomes but none from \ecoli. To test the frameshifting capacity of
this linker sequence, we initiate a biological experiment.

We create a bifunctional fusion protein containing beta-galactosidase (\bgals) 
and \xylE.  Beta-galactosidase levels correlate to
levels of nitrophenol, which has a yellow color, while \xylE\ expresses
an enzyme that cleaves catechol.  These two proteins are joined by
the proposed linker sequence.  Research indicates that even in this
fused protein, both \bgals\ and \xylE\ should retain their functionalities.
As such, if the frameshift were to occur, one should observe both 
high levels of nitrophenol and low levels of catechol.

The biological construct for this experiment is in the process of being created.
Results should be available by mid-October.

\section{Discussion}
First, we note that our model is quite robust: Minor changes
in $\theta_{\rm{sp}}$ and initial displacement do not affect
frameshifting significantly.
\autoref{prfB:sens} illustrates the error-free rate as
a function of species angle and initial displacement. As noted,
the frameshift holds over quite a wide range. Though these initial
parameters are necessary (\autoref{section:parameters}), they are not
crucial to the true interplay within the model between free energy and
translational efficiency.

In addition, our model agrees with biological evidence in that
ribosomal proteins express much more highly than genes in
general~\cite{rpoS:process} (\autoref{section:riboproteins}). This
concrete evidence, in addition with others presented in relation to
\prfB\ and bGH form strong evidence that our model can accurately
predict protein synthesis efficiency with its metrics (\autoref{section:metrics}).

Our correlation in running bovine growth hormone again indicate this
crucial relationship between translational
efficiency~\cite{schoner:bgh} and the metrics of our model. With
continual refinement, our model can then can significantly increase
the speed at which geneticists and biologists can obtain valuable
information when synthesizing commercially or medicinally useful
compounds without laboriously working through biological experiments
especially in the commercial value of bGH in this instance.

However, one immense limitation of our model lies in its concentrated
focus upon translation. However, 

Overall, the benefits from an accurate computational model in
modeling translation accrue when biologists need a large amount of
runs without the stopgap of experimentation, in vivo or in
vitro. Naturally, our model then possesses the power to imbue
commmercial or scientific pursuits with the wind beneath their sails
they need in a rapidly innovating and developing field of genetics
with enormous potential and applications.

\phantomsection
\addcontentsline{toc}{section}{References}
\begin{singlespace} \bibliography{wizards} \end{singlespace}
\end{document}
