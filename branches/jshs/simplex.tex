\documentclass[10pt,twocolumn]{article}
\usepackage{candor}
\begin{document}

\title{{\bf Bovine Growth Hormone}}
\author{{\sc W.H.Y. Lian, V.X. Bhattcharya, and D.R. Vitek}}
\date{{\sc \today}}
\maketitle

\section{What We Did}
\label{whatwedid}

Schoner et al.~\cite{schoner:bgh} devised variations of the bovine growth hormone (bGH) sequence.
We executed them voluptuously through the current model, refining it as we went along by including
a more precise metric of yield through deviation and incorporating ribosomal pausing.

\subsection{Measuring Deviation}
\label{whatwedid:lbd}

In theory, a sequence with optimal yield shows a displacement plot very near the predicted reading frame,
usually the horizontal axis. As such, deviation from the predicted axis offers a measure of ``yield."
We propose the formula
\begin{equation}
    \label{eqn:lbd}
    \sigma = \sqrt{\frac{\sum_i \left(x_i - \beta_i\right)^2}{N}}
\end{equation}
where $\beta_i$ is the predicted reading frame at codon $i$, $x$ is the displacement at codon $i$,
and $N$ is the total number of codons to embody this yield. Usually $\beta_i = 0$ unless a biologically verified frameshift exists. For example, for \prfB\ $\beta_i = 2$ for all $i \geq 25$ because \prfB\ frameshifts at codon 25 uga
and the model represents a frameshift with +2 displacement.

Thus, $\sigma$ is bounded by $[0,\infty)$, with zero corresponding to high yield and infinity to low yield.
Although no single equation can tie it with biological verification, running a small library of gene sequences
against the new yield anecdotally confirms a correlation between deviation and biological yield.

\subsection{Ribosomal Pausing}
\label{rp}

Research~\cite{gowri:pause} suggests that ribosomes often pause at distinct problem codons repeatedly
across translations given the same gene. Up until investigating bGH, the model did not
account for ribosomal pausing, simply letting every ribosome to translate boundlessly from start to finish. With
bGH, however, the model imposes a random cycle limit within the range $[150,250]$, random in order to accurately
determine the yield when running a sequence repeatedly. Upon reaching the limit the simulated ribosome
halts pausing and sadly dissociates.

\section{Analyzing bGH}

Through these metrics, the model yielded startling results.

\subsection{Polar Plots}

The polar plots for all sequences of bGH look similar after approximately twenty codons.  
However, nota bene, \emph{pcZ101} and \emph{pcZ105} differ in the first few codons compared to
their siblings (Figure~\ref{bgh:high}).
While the other sequences jump directly to 15$\degree$, pcZ101 and pcZ105 start at approximately $-30\degree$
(Figure~\ref{bgh:low}),
the species angle, a behavior common with many high-yield proteins including ribosomal proteins that do not
frameshift. Indeed, polar plots of high yield, non-frameshifting proteins obey the attraction to the species
angle, cf. Figure 1 of Ponnala, et al.~\cite{lalit:mechanics}.

For example, \emph{aceF} and \emph{rplE}, both high-yield proteins, have polar plots very close to $\theta = \varphi_{sp}$.
Even \prfB\ begins at $\theta = -90\degree$ and cycles clockwise through the species angle before
stabilizing around its final phase. The proximity to the species angle and its commonality to
other high-yield proteins then offers a measure for yield for bovine growth hormone.


\subsection{Deviation}

Table~\ref{deviation} summarizes the results from calculating the deviation metrics 
for all bGH sequences. The highlighted low deviations for \emph{pcZ101} and {\it pcZ105}
indicate possibly higher yields, consistent with already present biological results~\cite{schoner:bgh}.
Determining the biological significance of this possible correlation exists as future research.

\begin{table}[tbp]
\begin{center}
    \begin{tabular}{lc}
        \toprule
        \textbf{Sequence} & $\mathbf{\sigma}$\\
        \midrule
        pcZ101 & 0.76719\\
        pcZ105 & 0.76710\\
        \midrule
        pcZ100 & 0.80020\\
        pcZ104 & 0.79729\\
        pcZ108 & 0.79241\\
        pcZ110 & 0.79760\\
        pcZ112 & 0.79419\\
        pcZ115 & 0.79794\\
        \bottomrule
    \end{tabular}
    \caption{Deviation for bGH Sequences with Sample Size~200}
    \label{deviation}
\end{center}
\end{table}

\subsection{Ribosomal Pausing}

Table~\ref{wichita} tabulates the number of ribosomal pauses for each bGH sequence
using the process defined in~\ref{rp}. The two high-yield sequences show higher numbers of pause
dissociations, contrary to intuition, and presents itself as a further topic for exploration.

\begin{table}[tbp]
\begin{center}
    \begin{tabular}{lc}
        \toprule
        \textbf{Sequence} & \textbf{Dissociations}\\
        \midrule
        pcZ101 & 0.480\\
        pcZ105 & 0.526\\
        \midrule
        pcZ100 & 0.494\\
        pcZ104 & 0.444\\
        pcZ108 & 0.450\\
        pcZ110 & 0.476\\
        pcZ112 & 0.454\\
        pcZ115 & 0.466\\
        \bottomrule
    \end{tabular}
    \caption{Probability of Pause Dissociations with Sample Size~500}
    \label{wichita}
\end{center}
\end{table}

\bibliography{wizards}

\onecolumn
\appendix
\section{Bovine Growth Hormone Polar Plots}

Note the trend toward $\theta = \varphi_{sp}$ in the Figure~\ref{bgh:high} that does
not exist in Figure~\ref{bgh:low}. For lack of space, low-yield sequences beside pcZ100 and pcZ104
are reduced in size due to their extreme similarity to \emph{pcZ100} and \emph{pcZ104} (Figure~\ref{bgh:low}).

\begin{figure}[htp]
    \centering
    \caption{High yield bGH sequences}
    \label{bgh:high}
    \subfloat[pcZ101]{\includegraphics[scale=0.7]{bgh/pcZ101}}
    \subfloat[pcZ105]{\includegraphics[scale=0.7]{bgh/pcZ105}}
\end{figure}

\begin{figure}
    \centering
    \caption{Low yield bGH sequences}
    \label{bgh:low}
    \subfloat[pcZ100]{\includegraphics[scale=0.7]{bgh/pcZ100}}
    \subfloat[pcZ104]{\includegraphics[scale=0.7]{bgh/pcZ104}}\\
    \subfloat[pcZ108]{\includegraphics[scale=0.3]{bgh/pcZ108}}
    \subfloat[pcZ110]{\includegraphics[scale=0.3]{bgh/pcZ110}}
    \subfloat[pcZ112]{\includegraphics[scale=0.3]{bgh/pcZ112}}
    \subfloat[pcZ115]{\includegraphics[scale=0.3]{bgh/pcZ115}}
\end{figure}
\end{document}