\documentclass[12pt, draft]{article}
\usepackage{candor, setspace}
\linespread 2
\numberwithin{equation}{section}

\begin{document}
\section{Introduction}

\section{Description of the Model}
\subsection{Free Energy}
\label{freeenergy}

Free energy in a cell, which arises from the hybridization between two
sequences of RNA, drives ribosomal translation~\cite{starmer}.

Research suggests the 16S tail of the ribosome hybridizes with its  mRNA~\cite{weiss};
from this, we can  calculate the free energy values for the interaction between the two strands of RNA.
Someone et al.~\cite{freier} propose a thermodynamic model for exactly this,
modelling the interaction in terms of \emph{doublets}, i.e. pairs of consecutive nucleotides.
Using this model, Freier et al.~\cite{freier} numerically calculated the free energy available
upon hybridization for all permutations of RNA doublets.

We use Freier et al.'s values to calculate a free energy value at each nucleotide.
First, the 13-base 16S tail of \ecoli\ hybridizes with the first 13 bases of a sequence, which
for a gene is a 12-base leader sequence and the first base of the start codon, to determine
the free energy associated with the first nucleotide.
Then, shifting one base pair at a time, we calculate the free energy for the entire sequence.
Since any free energy value greater than zero represents binding that would only take place if energy were added,
we set such values to zero.

% We need to introduce the concept of displacement somewhere; I'm not sure where, though -- Vivek

\subsection{A Deterministic Model for Displacement}

Ponnala et al.~\cite{lalit:mechanics} assumes a sinusoidal model for free energy, whence one can calculate
the magnitude and phase of ribosomal translation using arctan given the free energy signals once the memory model
also proposed by Ponnala et al. stores those values in a phasor memory structure,
where memory here is biologically equivalent to the displacement from the reading frame of the ribosome performing the translation
and the registers represent a snapshot of the free energy values at a given codon during translation. The magnitude and phase
of the ribosomal translation here represents the phasor Ponnala uses to structure the register contents and are used to generate
polar plots.

Ponnala et al.~\cite{lalit:embs} represents the cummulative phasor at codon $k$ as $\bvec{V} = Me^{i\theta}$
where $i$ is the imaginary constant, projecting the magnitude and phase onto the complex plane such that
we can model the polar plot for easier differentiation.
Differentiating this vector, we arrive at vector $\bvec{D}$ for instantaneous energy, interpreting this
vector as a force that acts on the ribosome, keeping the mRNA in a given reading frame.

The length of time that the force acts on the ribosome depends up the codon's tRNA availability at the A-site during translation.
Ponnala use a deterministic model to represent this: For each codon, a number
of ``wait cycles," dependent upon the rarity of the associated tRNA, correspond to the number
of times the force is allowed to increment displacement from the current reading frame.
In their deterministic model, the force acts for \emph{exactly} this set number of cycles, given a certain codon in a certain gene sequence.

\subsection{A Stochastic Model for Displacement}

However, in practice it is idealistic to assume that all sequences, especially those that frameshift, will be read in exactly the same manner every single time that a given mRNA strand passes through the ribosome.  In order to account for this essentially random 

\subsection{Frameshifts}

% Where the following concept should be first mentioned, I'm not sure -- Vivek

As mentioned earlier, a displacement of $x = 0$ corresponds to translation along the 0 reading frame,
while a displacement of $x = 2$ represents translation along the $+1$ reading frame.
Ponnala et al.~\cite{lalit:embs} prove that both $x = 0$ and $x = 2$ are stable points in displacement.

% Figure prfB:lalitdisp will be prfB on Lalit's model.  Figure prfb:polar is the polar plot for prfB. Obviously.  --Vivek

A sudden jump from approximately $x = 0$ to $x = 2$ is the first indication of a $+1$ frameshift; 
in essence, this jump suggests that the ribosome jumps one entire base pair in the mRNA sequence.
Figure~\ref{prfB:lalitdisp} shows the displacement plot---by this deterministic model---for \prfB, 
a gene known to have a programmed $+1$ frameshift at codon 26.

In conjunction with this characteristic plot, however, a $+1$ frameshift also shows a 240\degree
rotation of phase angle from the species angle.
Intuitively, a frameshift would only be sustained if the free energy signal aligns itself with the sudden jump in displacement.
Since a the free energy signal has a period of one codon, for a $+1$ frameshift---a one-nucelotide jump---the free energy signal
must undergo a phase shift of one-third of an entire period.  Figure~\ref{prfb:polar} indicates this shift in \prfB.

\subsection{A Stochastic Model for Displacement}

% Find an example of a gene with a screwed-up displacement plot under Lalit's old model. --Vivek

The gene \prfB\ shows a characteristic frameshift under the deterministic model: The displacement jumps to $x=2$.
Certain genes, however, demonstrate ambiguous behavior near $x = \pm 1$.
Coupled with the limitation that a deterministic model will output a fixed graph for a given sequence, 
we cannot clearly interpret whether this behavior indicates a tendency for a stable frameshift.
As such, we introduce probability into the model.

We propose that at each cycle in elongation, the ribosome must make a decision: stay in the current reading frame,
move to the $+1$ reading frame,
or move to the $-1$ reading frame.  We can further subdivide this choice into the individual wait-cycles.
At each wait cycle, the ribosome can choose one of the above possibilities or proceed to another wait-cycle without making a decision.

Let $abcd$ be a sequence of four nucleotides, with $abc$ in the
current reading frame and $bcd$ in a +1 reading frame.  Let $x$ be the
displacement as of the current wait cycle of the ribosome.  As the
incremental displacement approaches +1, the probability of choosing
codon $bcd$ should increase, and the probability of choosing codon
$abc$ should decrease.  We modeled this behavior using even powers of
nthe cosine and sine functions for $abc$ and $bcd$, respectively.  We
define $\omega$ as the weight, which will be directly proportional to
the probability.

\begin{equation}
  \omega_{abc} = \cos^{10}{\frac{x\pi}{4}} \text{ and } \omega_{bcd} = \sin^{10}{\frac{x\pi}{4}}.
\end{equation}

Here, $n$ represents the wait time for that codon and $N$ is the ordinal of the wait cycle. (For example,
$N=1$ on the first wait cycle, $N=2$ on the second, and so forth.)

Now we define a value $n_{abc}$ based on normalizing $N_{abc}$, the
number of wait cycles (which is itself inversely proportional to the
tRNA availability of codon $abc$).  We present a logical argument for
our normalization.

Assume that we know nothing about the mechanics behind frameshifts.
Intuitively, we would guess that the probability of frameshifting
after $N_{abc}$ wait cycles is $\frac{1}{2}$.  Let $P$ be the
probability of moving on to the next codon and staying in a +0 reading
frame.  Then we should have $1-\left(1-P\right)^{N_{abc}} =
\frac{1}{2}$.  But we already have argued that $P \propto
\omega_{abc}$, so we define $n_{abc}$ to be the constnt of
proportionality. That is, $n_{abc} = \omega_{abc} / P$.  Coupled
with that $\omega_{abc} \le 1$, we have

\begin{equation}
  n_{abc} \le \frac{\sqrt[N_{abc}]{2}}{\sqrt[N_{abc}]{2} - 1}.
\end{equation}

Mathematically, the probability of choosing a codon is then
\begin{align}
  \prod^N \left(1-\frac{\omega}{n}\right) \text{ where } \omega = \cos^{10}{\frac{x\pi}{4}}.
\end{align}
Here, $n$ represents the wait time for that codon and $N$ is the ordinal of the wait cycle. (For example,
$N=1$ on the first wait cycle, $N=2$ on the second, and so forth.)

\section{Applications}

\subsection{Bovine Growth Hormone}
Besides the frameshifting sequence \prfB, our model also shows a significant correlation in data with biological
bovine growth hormone data, sequenced previously in \ecoli~\cite{schoner:bgh}. For example, our model purports a relation
between the deviation from the zero reading frame---for non-frameshifters anyway---and protein yield, ultimately postulating
a relation between ribosomal efficiency and its ability to maintain its reading frame without wobbling. Schoner et al.
modified early codons of an initial bovine growth hormone sequence to produce a series of alternate bGH sequences,
two of which produced significantly higher yield than their compatriots. Those two, dubbed pcZ101 and pcZ105, procured significantly
lower deviation than pcZ104, pcZ108, and other bGH sequences, thus agreeing with the biological data from Schoner.
A model's accurate prediction of yield given a sequence would greatly benefit the work of biologists across the world,
aiding in verifying the yields of a large sequence of proteins especially when experiments on all of them proves too difficult
or costly. Therefore, bGH is an important cornerstone of our computational model's impact on biology.

\subsection{rpoS RNA Polymerase Protein}
Because our model easily enables taking large samples of frameshifts without complicated biological experiments,
we can employ the sample probability that a gene sequence will run correctly without any mistake frameshifts because
at high sample sizes it readily approximates the population probability, which we have shown with sequences such as
\prfB\ and ribosomal proteins contains a positive correlation to actual protein yield. In addition, research~\cite{rare:yield}
suggests we can remove trouble spots where mistake frameshifts often occur by replacing those codons and their
earlier neighbors with synonymous codons\footnote{Two codons are synonymous if and only if they produce the same amino acid.}.
Therefore, we wrote a locally optimal (greedy) algorithm that finds trouble spots and replaces them with random sequences
of synonymous codons, running those permutations through our model for their sample probability and choosing the
sequence with the highest yield for each local trouble spot. Hence, locally optimizing. One ribosomal protein with
low yield is \rpoS. Running \rpoS\ through our algorithm, we found 33 codon changes that produced a higher yield.
The final \rpoS\ is now entering a biological experiment. If it proves that it does indeed express higher levels
of \rpoS\ RNA polymerase, then our model represents a breakthrough in optimizing gene sequences for applications
such as commercial production. By optimizing potentially useful gene sequences as past scientists have done with
insulin, our model aids the production medicinally and comercially important proteins by replacing a tedious and slow
biological process with a computational model, again helping people across the world.

\footnotesize
\begin{singlespace}
  \bibliography{wizards}
\end{singlespace}
\end{document}
