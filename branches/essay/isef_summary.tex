\documentclass[twocolumn, article, oneside]{memoir}
\usepackage{candor, verbatim, abstract}
\usepackage[charter]{mathdesign}
\renewcommand{\abstractname}{}
% Suppress memoir's fancy headings.
\pagestyle{plain}
\linespread{1.2}

\hyphenation{frame-shifts gen-bank da-ta-base hy-bri-di-za-tion}

\newcommand{\BWFtitle}[1]{A Computational Model for Translational
  Efficiency and Frameshifts in #1{Escherichia coli} Using a Genetic Signal
  Processing Approach}
\newcommand{\BWFauthors}{Hao Lian, Vivek Bhattacharya, and Daniel
  R. Vitek}

\usepackage[final, colorlinks=true, linkcolor=BWFBlue,
  citecolor=BWFGreen, urlcolor=BWFRed, pdftitle={\BWFtitle{}},
  pdfauthor={\BWFauthors}, pdfcreator={The Frameshift Kids},
  pdfstartview={FitH}]{hyperref}

\author{\BWFauthors}
\title{\BWFtitle{\emph}}
\renewcommand{\maketitle}{%
  \Large \textbf{\thetitle} \par \theauthor \par
}

\begin{document}
\section{Question}
To what extent can we successfully model ribosomal translation within
\ecoli\ computationally and how can that model help predict
translational efficiency and frameshifts, both programmed and
accidental?

\section{Hypothesis, Problem, Goals}
We hypothesize that, in general, a computational model for ribosomal
translation can be developed to provide a method to estimate translational
efficiency and predict the location of programmed frameshifts.  We further
hypothesize that the genetic sequence in the messenger RNA codes for a 
specific signal which is decoded by the ribosome.  This signal can be 
approximated by studying the free energy of hybridization (bonding) between
a particular part of the ribosome--the 16S tail---and the messenger RNA itself.

We are part of an ongoing research project
investigating the application of bioinformatics
and genetic signal processing to better understand how
information is encoded to and decoded from nucleic acids.  The particular
focus of the current research on ribosomal translation is within
bacteria.  Previous researchers~\cite{lalit:mechanics}
have developed a deterministic model
of translational reading frame by studying the
programmed frameshift present in the \prfB\ gene of \ecoli.  The focus
of our studies was to improve the model and apply its vatic powers.

Specifically, the goal of this research was to develop a 
\emph{stochastic} model for ribosomal translation.  We first engage in
a mathematical study of the original model, and we propose a method
to add randomness and stochasticity into the model to better simulate
the cellular environment.  From the fundamental principles of this
proposed model, we then develop two metrics---one to estimate the
location and frequency of frameshifts, and another one to estimate the
translational efficiency of a certain mRNA.  In addition, this research
also presents a genetic algorithm that would improve parameter estimation
for this model.

This research is also designed to test the validity of our model
by comparing computational results with biological ones published in previous
experimental papers.  Test sequences are also proposed.


\section{Procedures}
We used Matlab and Perl to extend a previous computational model. We
collected all the data through the Matlab graphical user interface,
which represented the output from the Matlab and Perl programs.

\subsection{Analysis}

\phantomsection
\addcontentsline{toc}{section}{Bibliography}
\bibliography{wizards}
\end{document}

% This is for ispell. Do not delete. --Hao
% LocalWords:  abcd abc bcd sp pcZ riboproteins disp aag guu efr ikemura jbsb
% LocalWords:  kane sd xray starmer aa TAV uga NCBI's Genbank lacZ hui UAA TAVs
% LocalWords:  Ecogene schoner kozak
