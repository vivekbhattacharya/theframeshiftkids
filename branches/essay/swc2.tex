\documentclass[11pt]{article}
\usepackage{candor, setspace, abstract, times, verbatim, setspace}
\hyphenation{frame-shifts gen-bank da-ta-base hy-bri-di-za-tion}

\newcommand{\BWFtitle}[2]{A Computational Model for Translational
  Efficiency and Frameshifts in #1{Escherichia coli} Using a Genetic Signal
  Processing#2Approach}
\newcommand{\BWFauthors}{Vivek Bhattacharya, Hao Lian, and Daniel
  R. Vitek}

\usepackage[draft, colorlinks=true, linkcolor=BWFBlue,
  citecolor=BWFGreen, urlcolor=BWFRed, pdftitle={\BWFtitle{}{ }},
  pdfauthor={\BWFauthors}, pdfsubject={Genetics}, pdfcreator={The
    Frameshift Kids}, pdfkeywords={bioinformatics,
    genetics,biology,frameshifts,ecoli,prfB,bGH,ecoli,translation},
  pdfstartview={FitH}]{hyperref}

\usepackage{titlesec}
\titleformat{\section}{\bfseries}{\thesection.}{.5em}{\MakeUppercase}
\titleformat{\subsection}{\bfseries}{\thesubsection.}{.5em}{\normalsize}
\titlespacing*{\section}{0pt}{\baselineskip}{*0}
\titlespacing*{\subsection}{0pt}{\baselineskip}{*0}
\titlespacing*{\subsubsection}{0pt}{\baselineskip}{*0}

\author{\sc{\BWFauthors}\\
William G. Enloe High School\\
Raleigh, NC}
\date{{\sc \today}}
\title{\bf{\BWFtitle{\emph}{ }}}

\linespread{2}
\begin{document}
\maketitle

\newpage
\section*{Acknowledgments}
We would like to thank our mentors, Dr. Donald L. Bitzer,
Distinguished University Research Professor; Dr. Mladen A. 
Vouk, Head of the Computer Science Department; and Dr. Anne-Marie
Stomp, Associate Professor of Forestry, for their advice, suggestions,
and support.  We offer our gratitude to North Carolina State University
for offering us use of their computational and laboratory facilities.  
Thank you to Dr. Fred Breidt and Robert Snyder for helping us design
and run experiments to test our predictions.  Thanks to Dr. Lalit
Ponnala for guiding us through his research, which formed the basis
of our work.  Thanks to Scott Vu for using our model for his work and offering
us his own comments as well.  In addition, thanks to our parents for their
support throughout this process.

% Title, table of contents, and abstract start things off.

\newpage
\begin{onehalfspace}
\tableofcontents
\end{onehalfspace}
\newpage

\begin{center}
\noindent\BWFtitle{\emph}{~}

Vivek Bhattacharya, Hao Lian, and Daniel Vitek

William G. Enloe High School, Raleigh, NC

Teacher Coordinator: Mr. Brian Wood

Mentors: Dr. Donald Bitzer, Dr. Anne-Marie Stomp, and Dr. Mladen Vouk
\end{center}
\begin{onecolabstract}

In genetics, \ecoli\ is used as an expression system to commercially
produce proteins. However, sequence-dependent features, such as rare
codons and codon biases, affect translational efficiency. To tackle
this problem, we proposed a stochastic model to computationally
estimate translational efficiency and predict frameshifting, uniting
ideas from biological literature and developing two predictive
metrics. We ran our model on 4364 sequences from the \ecoli\ genome
and found over 90\% of them to have predicted high yields; moreover,
the model predicts ribosomal proteins to translate at even higher
rates---both results that concur with experimental evidence. We
investigated a set of eight sequences of bovine growth hormone, and
the model correctly determined the translation efficiency for seven.
We then examined variations of \prfB, a gene with a programmed
frameshift; the model grouped these sequences into two general
categories of high and low yield, consistent with experimental
results. Successful development of a computational model for
translational efficiency implicates itself in optimizing recombinant
protein yield in multiple fields, including commercial protein
synthesis.
\end{onecolabstract}
\newpage

\section{Introduction}
We are part of an ongoing research project
investigating the application of bioinformatics
and genetic signal processing to better understand how
information is encoded to and decoded from nucleic acids.  The particular
focus of the current research on ribosomal translation is within
bacteria.  Previous researchers~\cite{lalit:mechanics}
have developed a deterministic model
of translational reading frame by studying the
programmed frameshift present in the \prfB\ gene of \ecoli.  The focus
of our studies was to improve the model and apply its vatic powers.

\citet{kozak05} and \citet{kane95} studied the impact
of sequence-dependent features, especially with regard to codon bias 
and rare codon usage, on translational efficiency.  The importance of 
these features is especially pronounced in the synthesis of recombinant 
proteins~\cite{sorensen05}.  In addition, secondary structure problems 
during protein folding can decrease translational 
efficiency~\cite{kozak05}.  However, scientists do not fully
understand the specific connections between efficiency and these
sequence-dependent factors, as even eliminating troublesome factors
does not always increase efficiency. The goal of our studies is then to
advance the development of a computational method that identifies mRNA
sequence changes to improve protein yield. If
successful, this work streamlines gene sequence optimization in the
production of recombinant proteins. The model also addresses the
significant challenge of creating cell lines
that synthesize proteins at commercial yields in fields from
medicine to agriculture. In addition, a successful model implies a mechanism by
which molecular biologists test the stable maintenance of
elongation.

Prior to our work, \citet{lalit:jbsb} created a deterministic model of
frameshifting based on the hybridization between the 16S rRNA tail and
mRNA nucleotides.  The periodicity of this
signal~\cite{lalit:jbsb} suggests that a force emerges from the free
energy of hybridization and acts as a mechanism to
stabilize reading frame.  We discuss limitations of a deterministic 
model in \autoref{stochastic}.

We assume that, at each cycle of ribosome translation, environmental
noise results in the ribosome's imprecise alignment
with the next codon to be translated.  To capture this idea, we
created a new metric of efficiency: the total deviation from the
intended reading frame.  We found this metric
roughly correlate with translational
efficiency (\autoref{section:deviation}).  Throughout this paper, we use this and other metrics to
optimize tRNA availabilities in order to distinguish between
high and low efficiency sequences. We also use our new
metric to optimize the performance of a translationally regulated gene
and to test the robustness of our model by running  ribosomal proteins and the
\ecoli\ genome.

\section{Prior Model}
\subsection{Free Energy}
\label{freeenergy}

Hybridization between two RNA sequences changes the free energy in a
cell, occurring between the rRNAs, the mRNA, and the tRNAs~\cite{starmer}.
\citet{sd} observed that the $3'$ end of the 16S rRNA is complementary to a sequence found 
directly upstream of the start codon of many prokaryotic mRNAs; they hypothesized that RNA hybridization 
plays an important role in translation initiation, later
experimentally confirmed~\cite{hui,jacob}.

\citet{weiss87} subsequently observed that changes in the 16S tail can 
significantly change the frequency of frameshifting in \prfB, an \ecoli\ gene 
known to frameshift at the 25$^\textrm{th}$ codon.  These results suggest that 
the 16S tail is positioned to interact with the mRNA during
translation and elongation. 
\citet{xray} supported the spatial accessibility of the 16S tail with the mRNA with 
X-ray crystallography data.

\citet{freier} proposed a thermodynamic model for calculation of free energy values,
modeling the hybridization of permutations of pairs of consecutive RNA nucleotides.

\subsection{Deterministic Model}
\citet{lalit:mechanics} assumed a sinusoidal model of free energy against base-pair position 
because a Fourier transform of free energy indicated a strong component with a period of one codon.  
In their model,
free energy projects onto magnitude and
phase through a memory model that stores three values
in a phasor, a concept from physics. 
\citeauthor{lalit:mechanics} represented this cumulative energy phasor
at codon $k$ as $\bvec{V} = Me^{i\theta}$, where $i$ is the imaginary
constant, from which they calculate the magnitude and phase, modeled
on a polar plot, via
trigonometry. Differentiating the energy phasor
with respect to distance along the mRNA strand gives 
vector $\bvec{D}$, the force assumed by the model~\cite{lalit:mechanics}
to act on the ribosome to keep the mRNA in the reading frame.
 
The time that the force acts on the ribosome depends upon
the tRNA availability associated with the codon at the A-site 
and interactions between said tRNA and the ribosome~\cite{phelps}.
\citeauthor{lalit:mechanics} represent this with a deterministic model: For each codon,
the number of ``wait cycles," a function of the rarity of the
tRNA, corresponds to the time the force can
increment displacement from the current reading frame.  In the
deterministic model, the force acts for \emph{exactly} this number
of cycles for a given codon. The model then
simulates hybridization between the
16S ribosomal subunit and a given mRNA strand: First, the 13-base 16S
tail of \ecoli\ hybridizes with the first 13 bases of a sequence,
which includes a 12-base leader sequence, to determine the free energy 
value of the first nucleotide of the coding sequence. The algorithm then iteratively calculates
the free energy for every nucleotide of the sequence~\cite{starmer}.

\subsection{Frameshifts}
\label{section:frameshifts}

\begin{cfigure}
  \parbox{0.4\linewidth}{
    \caption{Plots of~\prfB: Deterministic displacement}
    \label{prfB:deterministic:sub}
    \includegraphics[width=\linewidth]{prfB/deterministic}
  }
  \qquad
  \begin{minipage}{0.4\linewidth}
    \caption{Plots of~\prfB: Polar plot}
    \label{prfB:polar:sub}
    \includegraphics[width=\linewidth]{prfB/polar}
  \end{minipage}
\end{cfigure}

\citeauthor{lalit:mechanics} let a displacement of $x = 0$ correspond
to the zero reading frame and increments of \emph{two} to represent a
one-nucleotide change. For example, $x =2$ represents the +1 frame.
They prove that both $x = 0$ and $x = 2$ are
stable points, as expected.

A jump from approximately $x = 0$ to $x = 2$ in a span of only
base pair is then the first indication of a $+1$ frameshift; it
suggests the ribosome skips one entire base pair in the mRNA sequence.
\autoref{prfB:deterministic:sub} shows the displacement plot per this
deterministic model for \prfB, a gene with a unique programmed $+1$
frameshift, which exists at codon 25. In conjunction with this
characteristic plot, a $+1$ frameshift also displays a 
characteristic clockwise 120\degree\ phase angle rotation from the
species angle, the average phase of the free energy signals of a
number of verified \ecoli\ genes that stay in frame
\cite{lalit:mechanics}.  We interpret the free energy signal's
alignment with the sudden jump in displacement as a sustained
frameshift. Since the free energy signal has a period of one
codon~\cite{lalit:mechanics}, a $+1$ frameshift the free energy signal
must undergo a phase shift of a third of an entire period
(\autoref{prfB:polar:sub}).

\section{Computational Methods}

\subsection{Stochastic Displacement Model}
\label{stochastic}

As mentioned, the gene \prfB\ exhibits a programmed frameshift
by jumping in displacement to $x=2$.  Certain
genes, however, demonstrate ambiguous behavior near
$x = \pm 1$~\cite{lalit:mechanics}.  In the deterministic model, we lacked the
sensitivity needed to clearly discern a programmed
frameshift. Worse, the model may not show these unstable
behaviors at all. The nondeterministic
behavior of translation and the presence of noise limits the model's
ability to achieve better resolution. Therefore, we revamped the model to be
stochastic through the incorporation of sinusoidal probability.\footnote{
  The model's code is available online with detailed documentation.
}

At each wait-cycle in elongation, we propose the ribosome makes a
decision: stay in the current reading
frame, move to the $\pm 1$ reading frames, or proceed to the next cycle.
In addition, because the 
number of wait cycles is inversely proportional to the tRNA availability of 
the codon in the current reading frame (the
A-site)~\cite{lalit:mechanics, ikemura}, rarer codons force the 
ribosome to wait longer for the appropriate
aa-tRNA. \citeauthor{lalit:mechanics}, calculated these
values from existing research~\cite{ikemura}, which
related them to codon frequency. From this
body of work, we created a different algorithm to
calculate them (\autoref{section:parameters}).

Let $abcd$ be a sequence of four nucleotides, with $abc$ in the
current and $bcd$ in the +1 reading frame, and let $x$ be the
displacement of the current wait cycle of the ribosome.  As the
incremental displacement approaches +1, the probability of choosing
codon $bcd$ increases and the probability of choosing codon
$abc$ decreases.

We model this behavior using even powers of
cosine and sine functions for $abc$ and $bcd$, defining
$\omega$ as the \emph{weight} that is directly proportional to
the probability. It must meet these criteria:
If the ribosome lies completely and thus is stable in the
zero frame, then the probability
of staying in that frame is one.  Consequently, a ribosome
fully in the +1 frame has no chance of going to the zero
frame, hence the requirement for a period of two base pairs ($x=4$) in
these functions. Thus, we propose
\begin{equation}
  \omega_{abc} = \cos^{10}{\frac{x\pi}{4}} \text{ and } \omega_{bcd} =
  \sin^{10}{\frac{x\pi}{4}}.
  % Footnote
  \footnote{The cosine and sine functions are taken to the tenth power
    here. These parameters can change.}
\end{equation}

Suppose we are on a wait cycle at codon $abc$ with $N_{abc}$ total
cycles allocated. Let $P$ be the instantaneous probability of staying
in the current reading frame at the next move, which we know (above) is
proportional to the weight $\omega_{abc}$. Let $1/n_{abc}$ be
the constant of proportionality, implying $n =
\omega_{abc}/P$.\footnote{We can derive $n$ as follows:
  We know $\omega = cos^{10}(x\pi/4) \le 1$, implying $P \le 1/n$. Assume the probability of
  moving (1/2) is just as likely as the probability of not moving
  (1/2). Suppose there are $N$ wait cycles. Then $1 - (1 - \omega/n)^N
  = 1/2$, implying $n \le \sqrt[N]{2}/(\sqrt[N]{2} - 1)$. We also have
  $\lim_{N\rightarrow\infty}\max{n} = N/\ln{2}$. Thus, the
  probability of choosing a codon at a cycle is proportional to its
  TAV because $N$ is proportional to its TAV, which coincides with intuition.
}
Then the aggregate probability of choosing codon $abc$ after $K$ cycles is
the probability of \emph{not} failing to change the reading frame ($1 - P$)
at every cycle for $K$ cycles. Hence, 
\begin{equation}
  1 - \prod_{i=1}^K \left( 1 - \frac{\omega_i}{n} \right) \text{ where }
  \omega_i = \cos^{10}{\frac{x_i\pi}{4}}.
  % Footnote
  \footnote{The weight depends upon the frame choice in question.}
\end{equation}

\subsection{Consequences of the Stochastic Model}
The stochastic model introduced a concept into the model proposed by
\citet{lalit:mechanics}: The ribosome has a finite probability of
``choosing the wrong codon," in
essence going out-of-frame due to a high tRNA availability. This is definitely not a programmed
frameshift, which is when the force pushes the ribosome to an unstable
point and moves it quickly to the +1 reading frame to regain stability.
Programmed frameshifts take place over the span of just
one codon; the graph never approaches $x=2$ over multiple codons.

An incorrect codon choice can occur when displacement nears $x = \pm 1$.
Our model assumes that the 0 and the $\pm 1$ reading frame codons have a finite
probability of occupying the A-site in the ribosome. From the probability
equations, the ribosome increasingly tends to stabilize around
the +1 frame by accident.  Unlike a programmed frameshift, incorrect codon choice results from
the slow digression from the true alignment and occurs over a number of codons.
As the ribosome nears $x = \pm1$, the values for both the sine 
and cosine functions drop, thus increasing the chance of increasing
the time required for translation. This models the physical
behavior of ribosome's tendency to pause as it stabilizes upon an aa-tRNA in the A-site.

Choosing the wrong codon is a purely stochastic phenomenon; only through our new
model can we actually track the actions of the ribosome throughout
translation. From our physical conceptualization (\autoref{stochastic}),
we establish the potential to measure translational efficiency below.

\section{Analysis}
To test our computational model, we ran a number of
experiments to analyze sequences present or expressed in \ecoli.

\subsection{Measures}
\label{section:metrics}

As this model is stochastic, multiple runs (the sample size) must be analyzed.
As such, we propose two sample metrics for analysis.

\subsubsection{Error-Free Rate}
\label{section:efr}
When studying a
sequence with a programmed frameshift, \emph{error-free rate} measures the percentage of runs 
during which the ribosome chooses the correct codon
at \emph{every} juncture.  In the case of a known +1 frameshift, the ribosome must
choose the +1 frame at the frameshift codon and stay in the 0 frame before
and the +1 frame thereafter for the run to be error-free.

\subsubsection{Displacement Deviation}
\label{section:deviation}

We define \emph{displacement deviation} to be
\begin{equation}
    d = \sqrt{\frac{\sum_i (x_i - \beta_i)^2}{N}},
\end{equation}
where $\beta_i$ is the predicted reading frame at codon $i$, $x$ is
the displacement at codon $i$, and $N$ is the total number of codons
as a measure of the deviation of the sequence from the expected
reading frame.  Usually $\beta_i = 0$ unless a programmed frameshift
exists as in \prfB.  For example, for \prfB, $\beta_i = 2$
for all $i \geq 25$ because \prfB\ frameshifts at codon 25
\textsc{uga} and the model represents a frameshift with +2
displacement per \autoref{section:frameshifts}.

\subsection{\prfB\ and Related Sequences}
In the first test of our model, we investigated the gene
\prfB\footnote{
  \prfB\ encodes protein release factor 2, which enters
  the A-site and causes protein synthesis to terminate at the stop
  codons \textsc{uga} and \textsc{uaa}. Since \prfB\ itself requires 
  translation to continue past a stop codon, this sets up an elegant 
  autoregulatory mechanism. We obtained \prfB's nucleotide sequence from
  NCBI's Genbank database at \url{http://www.ncbi.nlm.nih.gov/} with accession
  number NC\_000913.
}, which has a programmed frameshift.
\citet{weiss87} conducted a number of experiments regarding
\prfB\ to test how mutations in the sequence affect the rate of
frameshifting.  They present a total of 35 sequences in their paper,
along with measures of translational efficiency.  We hypothesized that
genes found to frameshift at high rates by
\citeauthor{weiss87} should also show high error-free rates under
our model.

\subsection{Ribosomal Proteins}
We also hypothesized that displacement deviation
provides a suitable metric for translational efficiency: A lower
deviation should correspond to a more efficient sequence.
This is because if, at each wait cycle, the
ribosome fails to stabilize in a period of indecision, the probability
that it will change reading frames increases. The translational
process resolves this indecision by either stabilizing around the
incorrect reading frame or pausing. The former synthesizes an
incorrect primary structure or causes premature truncation due to a
out-of-frame stop codon further downstream; latter preserves fidelity at the
expense of speed. That is, while a single run may produce
a working primary structure of the protein, the sequence in question
ultimately is less efficient than a synonymous sequence with lower
aggregate probabilities, whether we measure this as the
error-free rate (\autoref{section:efr}) or displacement deviation
(\autoref{section:deviation}). In addition, a greater deviation from
the zero axis inherently implies a longer time for translation, 
since proximity to $\pm1$ increases the probability of waiting and not choosing
either codon. A high value thus reduces translational
efficiency, again contributing to the probability of
translational failure.

To this end, we performed two tests. In one test, we ran the model on
all the 4364 genes provided by the Ecogene database of
\ecoli\ genes.\footnote{\url{http://ecogene.org/}}
We predicted the majority to exhibit low
displacement deviations based on the assumption that evolution would select
for high translational efficiency. In another
test, we looked at ribosomal proteins, known to have high levels of
expression~\cite{rpoS:process}. For this set we predicted lower mean
deviation than those of the \ecoli\ gene sample.

\subsection{Bovine Growth Hormone}
Our final test involved the analysis of a set of sequences known to 
be translationally regulated.  \citet{schoner:bgh} created a set 
of eight sequences in an empirical effort to increase the yield 
of recombinant bovine growth hormone (bGH) synthesized in an 
\ecoli\ expression system. They
showed that four sequences were expressed 
at significantly higher levels than the others. We predicted that 
these should have lower displacement deviations than others.

\subsection{Parameters}
\label{section:parameters}
For all computational experiments in this report, we use a species
angle of $\theta_{\rm{sp}} = -30\degree$ and an initial displacement of 0.1
in accordance with \citet{lalit:mechanics}.
We also explore the effects of changing these parameters on the
error-free rate of \prfB.
A parameter that is much more difficult to estimate
is the tRNA availability vector (TAV).
\citeauthor{lalit:mechanics} estimated TAV by codon usage, 
surveying genes from \ecoli.
Although this assumption has experimental basis~\cite{ikemura}, 
no concrete evidence supports this assumption.
We chose to calibrate the TAV from existing experimental data, 
designing a genetic algorithm to improve these values based on 
bGH sequences while remaining close 
to the already determined values.

\subsubsection{Obtaining the tRNA Availability Vector}
We optimize the separation between the displacement deviations of the 
high-yield and low-yield sequences. First, we generate a list of 
randomly modified TAVs and calculate the ratio of the 
displacement deviation (\autoref{section:deviation}) for the four 
high-yield sequences to that of the other four. From there, we sort the 
modified vectors by this ratio and discard the worst half.
We repeated choose two of the remaining vectors based on rank, taking a weighted 
average to spawn a new vector.  After creating the 
next generation of a constant ``gene pool'' size, we delete 
the previous generation and repeat. After a fixed number of 
generations, the algorithm terminates and returns the most optimal vector.
This algorithm does not significantly alter the vectors; the average 
change to each value in the vector was merely 15.99\%. We use the new values
throughout the paper.

\section{Results}
\subsection{\prfB}

\begin{cfigure}
  \parbox{0.4\linewidth}{
    \caption{Plots of \prfB\ in a stochastic model: Displacement plot}
    \label{prfB:disp:sub}
    \includegraphics[width=\linewidth]{prfB/disp}
  }
  \qquad
  \begin{minipage}{0.4\linewidth}
    \caption{Plots of \prfB\ in a stochastic model: Sensitivity plot}
    \label{prfB:sens:sub}
    \includegraphics[width=\linewidth]{prfB/sensitivity}
  \end{minipage}
\end{cfigure}

The gene \prfB, as mentioned, is known
to have a programmed frameshift at the 25$^{\textrm{th}}$ codon.
\autoref{prfB:disp:sub} shows its displacement plot,
again with a distinctive jump at codon 25.
\footnote{Note that the polar plot is the same as \autoref{prfB:polar:sub}.
The new model does not alter the polar plot or the free energy calculations.}

Notably, the displacement plot does not reach $x=2$ over the span of
one codon, as the previous deterministic model predicted.  Rather, due to randomness, the
ribosome stabilizes the codon of the $+1$ frame in the A-site before actually reaching
a displacement of exactly 2.  The propensity to approach and stabilize at $x=2$
concurs with experimental evidence indicating the ribosome
stays in frame after the \prfB\ frameshift to produce full-length RF2.
\autoref{prfB:sens:sub} shows the error-free rate of \prfB\ as a function
of species angle and initial displacement and demonstrates the
robustness of our model (\autoref{section:discussion}).

\begin{cfigure}
  \caption{Comparison of experimental yield and error-free rate, 500 iterations}
  \label{weissboxplot}
  \includegraphics[width=0.5\linewidth]{histograms/weissbox}
\end{cfigure}

We used our computational model to determine whether a correlation exists between 
error-free rate and the yield of a reporter protein, $\beta$-galactosidase (\bgals).  
\citet{weiss87} investigated elements of the mRNA sequence that could change the frequency 
of frameshifting.  The frameshift site of \prfB\ was fused to the encoding sequence for 
\bgals\ so that \bgals\ activity was dependent on the frameshift occurring, 
thus serving as an indirect measure of frameshift frequency.  

Our metric, error-free rate, showed some ability to divide \citeauthor{weiss87}'s 35 constructs into 
two general categories: those with \bgals\ activity over 1650 whole-cell units and those with a 
lower activity.  This number is in relation to the original, unmodified \prfB\ sequence, 
which exhibited an activity of 6600 units.  These computations results suggest that 
error-free rate could be used to predict protein yield, although it lacks resolution.

Notably, \citeauthor{weiss87} did not maintain the amino acid structure of the polypeptides.  
This discrepancy could impact protein folding and half-life.  Although they
presented no evidence evaluating this idea, such a phenomenon would confound the interpretation 
of \bgals\ activity as a measure of frameshift frequency.

\subsection{\ecoli\ Genes}
\begin{cfigure}
  \parbox{0.4\linewidth}{
    \caption{Investigating a large sample of \ecoli\ genes: Displacement deviations}
    \label{ecoli:hist}
    \includegraphics[width=\linewidth]{histograms/everything}
  }
  \qquad
  \begin{minipage}{0.4\linewidth}
    \caption{Investigating a large sample of \ecoli\ genes: Comparison
      to ribosomal proteins}
    \label{ribosomal:comp}
    \includegraphics[width=\linewidth]{histograms/ribosomal}
  \end{minipage}
\end{cfigure}

\begin{cfigure}
\end{cfigure}

\autoref{ecoli:hist}\footnote{We truncate the histogram at three, excluding
  the outliers and less than 1\% of the sample.} is a histogram of the
deviation yields of 4364 genes of \ecoli, encompassing over 80\% of the entire
genome.  As predicted, 93.45\% of the genes had deviations in the 0
to 1 interval, agreeing with the assumption of natural efficiency.  The
average displacement deviation for these \ecoli\ genes is 0.4425 ($\sigma = 0.1537$), 
running 500 iterations per gene.

\subsection{Ribosomal Proteins}
\label{section:riboproteins}
To test further our idea that genes that translate efficiently should exhibit
low displacement deviations, we tested our model on ribosomal proteins, which
are known to be expressed at especially high levels.
The displacement deviation from $x=0$ for ribosomal proteins
is on average 0.2708 ($\sigma = 0.0884$) in comparison to the average of 0.4425 for our
large sample of \ecoli\ genes, which is significantly higher with a $p$-value of
0.0109 when performing a two-sample $t$ test on the means.

\subsection{Bovine Growth Hormone}
\label{section:bgh}

\begin{cfigure}
  \parbox{0.4\linewidth}{
    \caption{bGH: Displacement plot}
    \label{bgh:disp}
    \includegraphics[width=\linewidth]{bgh/all}
  }
  \qquad
  \begin{minipage}{0.5\linewidth}
    \centering
    \caption{bGH: Deviations with sample size~500}
    \label{bgh:deviation}
    \begin{small}
      \begin{tabular}{lccc}
  \toprule
  \textbf{Sequence} & $d$ & $\sigma(d)$ & Yield (\% bGH)\\
  \midrule
  pCZ101 & 0.5146 & 0.03313 & 30 \\
  pCZ105 & 0.5139 & 0.04181 & 34\\
  pCZ112 & 0.6612 & 0.03633 & 33\\
  pCZ115 & 0.6721 & 0.03792 & 32\\
  \midrule
  pCZ100 & 0.7107 & 0.01715 & $<$ 0.5\\
  pCZ104 & 0.7162 & 0.01433 & $<$ 0.5\\
  pCZ108 & 0.5912 & 0.05976 & 1.7\\
  pCZ110 & 0.7026 & 0.01966 & $<$ 0.5\\
  \bottomrule
\end{tabular}

    \end{small}
  \end{minipage}
\end{cfigure}

We investigated the concept of displacement deviation as a predictive parameter
for experimental yield using published expression data~\cite{schoner:bgh} for bovine growth hormone (bGH).
The focus of this research was to modify the bGH mRNA sequence to optimize
yield of the protein in an \ecoli\ expression system.

\citet{schoner:bgh} created a number of constructs, primarily
modifying the initial codons of a bovine growth hormone sequence. The
research found that sequences pcZ101, pcZ105, pcZ112, and pcZ115 have
high protein yields in comparison to the four other sequences. We
found these aforementioned four sequences in addition to pcZ108 to
have the least displacement deviation from $x = 0$
(\autoref{bgh:deviation}). \autoref{bgh:disp} shows the displacement
plots of all the bGH sequences on the same set of axes.

Therefore, pcZ108 exists as an outlier. \citeauthor{schoner:bgh}
postulates it to have erroneously a low protein yield, attributing it
to an experimental error due to its similarity to pcZ114. They believe
that the low protein yield is not due to translational effects, a
hypothesis that could explain why our model predicts a relatively high
translation rate. Excluding this outlier from our data, we find that
our data (\autoref{bgh:deviation}) fully agrees with hers. In
addition---and like \citeauthor{weiss87}---\citeauthor{schoner:bgh}
changed the sequence to encode a different amino acid sequence than
the other constructs. We can then attribute the low translational
efficiency to the interaction of the primary structure with the
ribosome or to protein stability, subjects beyond the scope of our
model.

\subsection{An Artificial Frameshifter}
\label{section:linker}
To verify our model's predictive ability, we focused on its ability to
predict programmed frameshifts. We designed a sixteen-codon sequence
that should frameshift in \ecoli\ into
the \textsc{aag} frame after crossing the sole uracil.  Figure~\ref{linker:sens}
shows the error-free rate of the linker sequence as a function of species
angle and initial displacement, which is as robust as \prfB.

\begin{cfigure}
  \parbox{0.4\linewidth}{
    \caption{Linker sequence: Plasmid construct}
    \label{linker:plasmid}
    \includegraphics[width=\linewidth]{linker/plasmid}
  }
  \qquad
  \begin{minipage}{0.4\linewidth}
    \caption{Linker sequence: Sensitivity plot}
    \label{linker:sens}
    \includegraphics[width=\linewidth]{linker/sensitivity}
  \end{minipage}
\end{cfigure}

Next, we performed a BLAST search, which found no similar sequences in 
\ecoli. Then, working with molecular biologists, we helped create a strategy to 
determine the efficiency of frameshifting using a fusion protein. 
Biologists will fuse the linker sequence to the $5'$ to the end of \xylE~\cite{fred1},
which codes for catechol oxidase, monitored with a 
colorimetric reaction.  A plasmid vector (\autoref{linker:plasmid})
containing the \xylE\ fusion and \lacZ, a second reporter, genes~\cite{fred2}
will be constructed.  The \lacZ\ and the \xylE\ fusion will be 
co-transcribed on a polycistronic mRNA, but translated
separately.  The \lacZ\ gene product, \bgals, will be expressed constitutively,
but the expression of \xylE\ will necessitate a frameshift in the linker sequence.
As such, standardizing the catechol oxidase to the \bgals\ activity
will serve as a measure of the  efficiency. The experimental
work is in progress.

\section{Discussion}
\label{section:discussion}

The purpose of our studies was to extend and improve the deterministic
model of \citeauthor{lalit:mechanics}, which gave a mechanistic
perspective of ribosomal movement during translation and a genetic
signal approach to translating mRNA sequences. It incorporated a
number of parameters, including
codon bias~\cite{ikemura} and rare codon usage~\cite{kane95}, known to
affect translation. The model also could predict frameshift locations
in the sequence, of value in sequence annotation (frameshift identification).
Our studies extended this predictive power to 
computationally predict translational efficiency.

The deterministic structure limited the model of
\citeauthor{lalit:mechanics} (\autoref{stochastic}).
We restructured it into a stochastic process to better reflect cellular
environment conditions that can affect translation. In essence, the
stochastic model paints a more realistic picture of the ribosome, a
machine that makes choices nondeterministically due to noise from the
cell environment.

In addition to the stochastic version, we also developed two metrics:
error-free rate (\autoref{section:efr}) and displacement deviation
(\autoref{section:deviation}). Error-free rate
provides a measure of an mRNA sequence's propensity for frameshifting by
measuring reading frame change frequency.  Investigating the constructs used by
\citet{weiss87}, our model roughly separated sequences into
those with high and low frameshift frequencies. However, the
output is not always clear, and we turn to
displacement deviation in these cases.

Displacement deviation (\autoref{section:deviation}), when calculated
for a large sample of \ecoli\ genes assumed to be translationally
efficient, correlated with over 90\% of the experimental data,
predicting these genes to be efficient. Moreover, the results
predicted that ribosomal proteins should be more efficient than
average, in accordance with experimental knowledge. Also, displacement
deviations for seven of eight bGH sequences~\cite{schoner:bgh}
correlated with further experimental data; one sequence was an outlier
explained in Schoner's research.

Since the model is preliminary, we must note these outliers.  In the
case of bGH, \citeauthor{schoner:bgh} experimentally determined pcZ108 to be
of low yield, but the model computationally predicts high yield.
Likewise, in the set of constructs used by \citeauthor{weiss87}, one
set of sequences had low experimental yields, but the model predicted high translational efficiency.  One
possible explanation of these discrepancies relates to the chemical
structures of the mRNA sequences and the polypeptides they
encode.  In both cases, \citeauthor{weiss87} and
\citeauthor{schoner:bgh} did not maintain the amino
acid structure while designing the sequences. If a protein-ribosome
interaction or post-translational instability resulted from these
changes, then such a change could significantly impact protein yield, an
indirect measure of translation. However, these effects on
translational efficiency are beyond the scope of our model.

We also plan to explore parameter estimation in future studies.
Analysis of \prfB\ and the
proposed linker sequence suggest our model is robust with respect to species angle
and initial displacement, but proper estimation of tRNA availability
poses a problem. \citet{lalit:mechanics} based these values solely on 
codon usage, but research~\cite{phelps} also suggests that tRNA shape
also determines overall stability and is known to affect
frameshift frequency. The
genetic algorithm (\autoref{section:parameters}) we created thus
provides an initial method for a coarse estimation. However, we did
not extensively investigate the effect of tRNA shape and thus we limited the
difference between our values and those computed by \citeauthor{lalit:mechanics}.

Currently, our model predicts translational efficiency with some
accuracy and can distinguish between high and low yield sequences. 
Using results from our linker sequence (\autoref{section:linker}), 
currently in progress, we will adjust the model to increase its 
predictive power and consequently its importance in the field of 
genetics.

\section{Conclusion}
\label{section:conclusion}

In this paper, we presented a method, based on our newly
developed stochastic model and derived metrics that roughly predicts
translational efficiency.  More research is needed to
experimentally validate the model and improve parameter
estimates, especially those of TAV values. Despite these
shortcomings, the ability of our model to discern between efficient
and non-efficient sequences with respect to translation 
indicates future potential in the field of
recombinant biotechnology in the near future, providing a
computational, cost-effective method for gene construct
design when using \ecoli\ as an expression system.  In addition, 
we may be able to extend our model to other prokaryotic species with 
similar ribosomes.  Having a computational model based on the biological
mechanism of translation will further our understanding of the
translational process and be practical for the production of
recombinant proteins at a commercially useful scale.

\phantomsection
\addcontentsline{toc}{section}{References}
\begin{singlespace} \bibliography{wizards} \end{singlespace}
\end{document}

% This is for ispell. Do not delete. --Hao
% LocalWords:  abcd abc bcd sp pcZ riboproteins disp aag guu efr ikemura jbsb
% LocalWords:  kane sd xray starmer aa TAV uga NCBI's Genbank lacZ hui UAA TAVs
% LocalWords:  Ecogene schoner kozak

